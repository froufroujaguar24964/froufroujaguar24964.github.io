% !TeX spellcheck = en_US
\documentclass[11pt, a4paper]{article}

% Set the title of the current document to be produced.
\newcommand{\doctitle}{Assignment II}
% Command for the due date of the homework.
\newcommand{\duedate}{\color{rltblue}{\faCalendarCheckO { }Due date: August 16th, in class \faCalendarCheckO	}}
\newcommand{\myname}{\color{rltred} {}\textbf{Chaganti}  \textbf{Kamaraja}  \textbf{Siddhartha} }
\newcommand{\rollnumber}{\color{rltred}{}\textbf{EP20B012} }
%------------------------------------------------------------
% Import commands for both teacher and course information.  | 
% NOTE: Change your teacher and course info in these files. |
%------>------>------>------>------>------>------>------>-->|
%-------------------------------------------------
% Teacher-specific commands                      |
%---------------                                 |
%-> Instructions: change your teacher info here. |
%------->------>------>------>------>------>---->|
%
\newcommand{\instructor}{James Libby}
\newcommand{\office}{HSB 116A}
\newcommand{\hours}{By appointment}
\newcommand{\phone}{044 - 2257 4885}
\newcommand{\college}{IIT Madras}
\newcommand{\email}{Libby@iitm.ac.in}
\newcommand{\faculty}{Professor}
\newcommand{\department}{Physics}
                              %|
%-------------------------------------------------
% Course-specific commands                       |
%---------------                                 |
%-> Instructions: change your course info here.  |
%------->------>------>------>------>------>---->|
%
\newcommand{\semester}{July-Nov 2022}
\newcommand{\csection}{00001 \& 00002}
\newcommand{\ponderation}{2-4-3 (Theory-Lab-Homework)}
\newcommand{\coursetitle}{High Energy Physics}
\newcommand{\coursenumber}{PH5211}
\newcommand{\prerequisite}{All porgram courses semesters 1-4}
                               %|   
%
%------------------------------------------------------------
%-- Import packages and custom command definitions.          |
%------>------>------>------>------>------>------>------>-->|
%----------------------------------------------------
% The following is a list of LaTeX packages imports |
%------->------>------>------>------>------>---->---|
%

% The margins at the bottom of the page has been reduced.
% this allows for a slim footer.
\usepackage[left=1in,right=1in,top=1in,bottom=0.7in]{geometry}
% Original size:
%\usepackage[inner=1.5cm,outer=1.5cm,top=1.5cm,bottom=.5cm,margin=1in]{geometry}
\usepackage[
    colorlinks,
    pagebackref,
    pdfusetitle,
    urlcolor=blue,
    citecolor=blue,
    linkcolor=blue,    
    plainpages=false]
{hyperref}            
% ftp://ftp.dante.de/tex-archive/fonts/bbding/bbding.pdf
%https://ctan.math.illinois.edu/fonts/bbding/bbding.pdf
\usepackage{fancyhdr, lastpage, bbding, pmboxdraw}
\usepackage{fancyvrb}
\PassOptionsToPackage{usenames,dvipsnames}{xcolor}
\usepackage{acronym}
\usepackage{amsthm}
\usepackage{caption}
\usepackage{xcolor}
\usepackage{enumitem}
\usepackage{tabularx}
\usepackage{sectsty}
\usepackage{amssymb}
% pifont package doc at: https://ctan.math.ca/tex-archive/macros/latex/required/psnfss/psnfss2e.pdf
% pifont is used to define custom list and style list items using the \ding command. 
\usepackage{pifont} 
% bclogo used for making a colored box for notes. 
% @see: https://ctan.org/pkg/bclogo?lang=en
\usepackage[tikz]{bclogo} 
\usepackage{titlesec}  
\usepackage[open,openlevel=1]{bookmark}

%-- @see http://ctan.sharelatex.com/tex-archive/fonts/fontawesome/doc/fontawesome.pdf
% Font Awesome  http://ctan.math.washington.edu/tex-archive/fonts/fontawesome5/doc/fontawesome5.pdf
% https://muug.ca/mirror/ctan/fonts/fontawesome5/doc/fontawesome5.pdf
\usepackage{fontawesome5}
\usepackage{fontawesome}
%---------------------------------
% ==== Font setup.
% Load any of the following fonts.
%---------------------------------
%\usepackage{lmodern}
%\usepackage{mathptmx}
%\usepackage{times}
%\usepackage[sc]{mathpazo} % Palatino font.
%\linespread{1.05} % Palatino needs more leading (space between lines)
\usepackage{tgbonum} % For Bonum/Bookman font.
\usepackage[utf8]{inputenc}
\usepackage[T1]{fontenc}
%---------------------------------
\usepackage{booktabs} 

\pagestyle{empty}
\usepackage{graphicx}
\usepackage{multicol}
\usepackage{blindtext}  
\usepackage{vhistory} % for making a table for the revision history.
                                  %|  
%--------------------------------------------------------
%--> \customhrule: makes a customized rule whose width  | 
%                  should be passed as parameter.       |
%--------------------------------------------------------
\newcommand{\customhrule}[1]{
	\rule[1.4pt]{\linewidth}{#1}
}
%------------------------------------------------------
%--> \doublerule: makes a double rule.                |
%------------------------------------------------------ 
\newcommand{\doublerule}[1][.4pt]{
	\noindent
	\makebox[0pt][l]{\rule[.7ex]{\linewidth}{#1}}%
	\rule[1pt]{\linewidth}{#1}\par} 
%===== Custom Ruler commands  ==================
\renewcommand{\headrulewidth}{1pt}
\renewcommand{\footrulewidth}{0.4pt}

% Disable spaces between list items in a labeled list.
\setlist{noitemsep}
 
%-------------------------------------------------------------
%= The followig are declaraions of custom Lists              =
%-------------------------------------------------------------
%
%======= Green rectangles list =======================
% \Rectangle from bbind
\newlist{greenrectangles}{itemize}{4}
%\setlist[greenrectangles]{topsep=4pt,partopsep=0pt,itemsep=3pt,parsep=0pt,labelindent=0.5cm,leftmargin=*}
\setlist[greenrectangles]{itemsep=5pt,parsep=0pt,topsep=4pt,partopsep=3pt}
\setlist[greenrectangles,1]{font=\color{darkred},label={\color{darkgreen}{\Rectangle}}}

%======= Alphabetical  list =======================
\newlist{alphalist}{enumerate}{9}
\setlist[alphalist]{topsep=4pt,partopsep=0pt,itemsep=3pt,parsep=0pt,labelindent=0.5cm,leftmargin=*}
\setlist[alphalist,1]{label=\textbf{\alph*)}}
%======= Non-numbered list =======================
\newlist{itemizedlist}{itemize}{9}
\setlist[itemizedlist]{topsep=4pt,partopsep=0pt,itemsep=3pt,parsep=0pt,labelindent=0.5cm,leftmargin=*}
%\setlist[itemizedlist,1 ]{label=\textbf{\alph*)}}

%======= Arrowed list =======================
\newlist{arrows}{itemize}{4}
\setlist[arrows]{topsep=4pt,partopsep=0pt,itemsep=3pt,parsep=0pt,labelindent=0.5cm,leftmargin=*}
\setlist[arrows,1]{font=\color{darkred},label={\HandRight}}

%======= Bordered square list =======================
% Colorize the selected symbol? 
% ❏
\newlist{borderedsquare}{itemize}{4}
\setlist[borderedsquare]{topsep=4pt,partopsep=0pt,itemsep=3pt,parsep=0pt,labelindent=0.5cm,leftmargin=*}
\setlist[borderedsquare,1]{label=\ding{111}}

%======= Filled, curved arrow list =======================
\newlist{curveddarrow}{itemize}{4}
\setlist[curveddarrow]{topsep=4pt,partopsep=0pt,itemsep=3pt,parsep=0pt,labelindent=0.5cm,leftmargin=*}
\setlist[curveddarrow,1]{label=\small\faMarker}

%======= Colored pen list ======================= 
\newlist{coloredPen}{itemize}{4}
\setlist[coloredPen]{topsep=4pt,partopsep=0pt,itemsep=3pt,parsep=0pt,labelindent=0.5cm,leftmargin=*}
\setlist[coloredPen,1]{font=\color{darkred},label=\small\faMarker}

%======= Objectives list ======================= 
% ➠
\newlist{objectives}{itemize}{4}
\setlist[objectives]{topsep=4pt,partopsep=0pt,itemsep=3pt,parsep=0pt,labelindent=0.5cm,leftmargin=*}
\setlist[objectives,1]{label=\small\ding{224}}

%======= Dark starred list ======================= 
% ✸
\newlist{filledstarlist}{itemize}{4}
\setlist[filledstarlist]{topsep=4pt,partopsep=0pt,itemsep=3pt,parsep=0pt,labelindent=0.5cm,leftmargin=*}
\setlist[filledstarlist,1]{label=\small\ding{88}}

%======= Dark-bordered empty circle list ======================= 
% ❍
\newlist{emptyCircleList}{itemize}{4}
\setlist[emptyCircleList]{topsep=4pt,partopsep=0pt,itemsep=3pt,parsep=0pt,labelindent=0.5cm,leftmargin=*}
\setlist[emptyCircleList,1]{label=\small\ding{109}}

%======= Filled right arrow list ======================= 
% ➤
\newlist{filledRightArrowList}{itemize}{4}
\setlist[filledRightArrowList]{topsep=4pt,partopsep=0pt,itemsep=3pt,parsep=0pt,labelindent=0.5cm,leftmargin=*}
\setlist[filledRightArrowList,1]{label=\small\ding{228}}

%======= Numbered list: non-filled circle list ======================= 
% ➀
\newlist{numberedEmptyList}{itemize}{9}
\setlist[numberedEmptyList]{topsep=4pt,partopsep=0pt,itemsep=3pt,parsep=0pt,labelindent=0.5cm,leftmargin=*}
\setlist[numberedEmptyList,9]{label=\ding{182}}

%======= Right hand pointing list =======================
\newlist{rightHandPointingList}{itemize}{4}
\setlist[rightHandPointingList]{topsep=4pt,partopsep=0pt,itemsep=3pt,parsep=0pt,labelindent=0.5cm,leftmargin=*}
\setlist[rightHandPointingList,1]{font=\color{darkred},label={\HandRight}}

%----------------------------------------------------------------------
%=   The followig are custom colors declaraions                       |
%--  more colors codes can be found at: http://latexcolor.com/        | 
%-- usage: {\color{declared-color} some text}.                        |    
%  e.g.,: {\color{darkblue}{ This text will appear darkblue-colored}} |
%----------------------------------------------------------------------
\definecolor{darkblue}{rgb}{0,0,.6}
\definecolor{darkred}{rgb}{.7,0,0}
\definecolor{darkgreen}{rgb}{0,.6,0}
\definecolor{darkestred}{rgb}{.8,.1,0}
\definecolor{red}{rgb}{.98,0,0}
\definecolor{OliveGreen}{cmyk}{0.64,0,0.95,0.40}
\definecolor{CadetBlue}{cmyk}{0.62,0.57,0.23,0}
\definecolor{lightlightgray}{gray}{0.93}
\definecolor{vanierred}{RGB}{210,0,2}
\definecolor{darkestblue}{rgb}{0.0, 0.0, 0.55}
\definecolor{darkblue}{rgb}{0,0,.6}
\definecolor{darkred}{rgb}{.7,0,0}
\definecolor{darkgreen}{rgb}{0,.6,0}
\definecolor{darkestred}{rgb}{.8,.1,0}
\definecolor{red}{rgb}{.98,0,0}
\definecolor{OliveGreen}{cmyk}{0.64,0,0.95,0.40}
\definecolor{CadetBlue}{cmyk}{0.62,0.57,0.23,0}
\definecolor{lightlightgray}{gray}{0.93}
\definecolor{darkorange}{rgb}{255,140,0}
\definecolor{fluorescentyellow}{rgb}{0.8, 1.0, 0.0}
\definecolor{darkyellow}{rgb}{1,1,0.34}
\definecolor{lightyellow}{rgb}{1,1,0.6}
\definecolor{coolblack}{rgb}{0.0, 0.18, 0.39}
\definecolor{lightgray}{rgb}{.9,.9,.9}
\definecolor{darkgray}{rgb}{.4,.4,.4}
\definecolor{purple}{rgb}{0.65, 0.12, 0.82}
\definecolor{gray}{rgb}{0.4,0.4,0.4}
\definecolor{cyan}{rgb}{0.0,0.6,0.6}
\definecolor{dkgreen}{rgb}{0,0.6,0}
\definecolor{gray}{rgb}{0.5,0.5,0.5}
\definecolor{mauve}{rgb}{0.58,0,0.82}
\definecolor{lightblue}{rgb}{0.0,0.0,0.9}
\colorlet{punct}{red!60!black}
\definecolor{background}{HTML}{EEEEEE}
\definecolor{delim}{RGB}{20,105,176}
\colorlet{numb}{magenta!60!black}
\definecolor{coolblack}{rgb}{0.0, 0.18, 0.39}
\definecolor{forestgreen}{rgb}{0.0, 0.27, 0.13}
\definecolor{firebrick}{rgb}{0.7, 0.13, 0.13}
\definecolor{rltred}{rgb}{0.75,0,0}
\definecolor{rltgreen}{rgb}{0,0.5,0}
\definecolor{rltblue}{rgb}{0,0,0.75}
\definecolor{indigo}{rgb}{0.0, 0.25, 0.42}
\definecolor{jazzberryjam}{rgb}{0.65, 0.04, 0.37}
\definecolor{lava}{rgb}{0.81, 0.06, 0.13}
\definecolor{royalblue}{rgb}{0.0, 0.14, 0.4}
\definecolor{prussianblue}{rgb}{0.0, 0.19, 0.33}
\definecolor{prune}{rgb}{0.44, 0.11, 0.11}
\definecolor{cerisepink}{rgb}{0.93, 0.23, 0.51}
\definecolor{oxfordblue}{rgb}{0.0, 0.13, 0.28}
\definecolor{crimsonglory}{rgb}{0.75, 0.0, 0.2}
\definecolor{fireenginered}{rgb}{0.81, 0.09, 0.13}

%============================
% Commands for inserting colored text.
\newcommand{\bluetext}[1]{\textcolor{darkblue}{#1}}
\newcommand{\redtext}[1]{\textcolor{jazzberryjam}{#1}}

%=================================================================================================
% Command for styling tabled row header (left, center or right)
% Usage example: \thead{<Header text 1>} & \thead{<Header 2>} & \thead{<Header 3>} & \thead{<Header 4>} 
\newcommand*{\thead}[1]{\multicolumn{1}{l}{\bfseries #1}}	

%--------------------------------------------------
% ==== Doc header and footer setup.               |
%-------------------------------------------------- 
\renewcommand{\thefootnote}{\fnsymbol{footnote}}
\pagestyle{fancyplain}
\fancyhf{}
%- Disable the horizontal ruler in the header section.
\renewcommand{\headrulewidth}{0pt}
\rfoot{\fancyplain{}{page \thepage\ of \pageref{LastPage}}}
\cfoot{{\tiny{\college { } - { } \semester} }}
\lfoot{{\tiny{ \coursenumber -\coursetitle} }}
%- TODO: move the header content here.
\fancyfoot[RO, LE] {{\tiny{page \thepage\ of \pageref{LastPage} }}}
\thispagestyle{plain}
%------------------------------------------------------------

\newcolumntype{L}[1]{>{\raggedright\arraybackslash}p{#1}}
\newcolumntype{C}[1]{>{\centering\arraybackslash}p{#1}}
\newcolumntype{R}[1]{>{\raggedleft\arraybackslash}p{#1}}

%-- Spacing commands ------ 
\newcommand{\vspbpara}{\vspace*{.09in}}    
\newcommand{\customvspace}{\vspace{.5cm}}    
\titlespacing{\section}{0pt}{12pt}{9pt}
%-----
\newcommand{\vtitlespacing}{\vskip 0.3cm}
\newcommand{\paragraphentry}[1]{\noindent \textbf{\Large \underline{#1}} }
\newcommand \VRule[1][\arrayrulewidth]{\vrule width #1}
\newcommand{\bkt}[2]{\left \langle #1 \middle| #2 \right \rangle}
\newcommand{\braketmatrix}[3]{\left \langle #1 \middle| #2 \middle| #3 \right \rangle}   
%
%---> Generate & inject metadata describing                |
%     the produced document                                 |
%--------------------------------------------------------------
%-- Set up the hyperref package.                              |
%-- Generate and inject metadate in the produced PDF document |
%------>------>------>------>------>------>------>------>-->---
 \hypersetup{pdfauthor={\instructor},%
    pdftitle={\coursenumber -- \coursetitle},%
    pdfsubject={\doctitle, Section \csection {} (\semester)},%
    pdfkeywords={\college,  \department},%
    pdfproducer={LaTeX},%
    pdfcreator={pdfLaTeX},
    bookmarks,
    bookmarksnumbered = true,
    bookmarksopen     = true,
    pdfpagelabels     = true,
    pdfstartview={XYZ null null 1.2}
}                                  %|
%------------------------------------------------------------

\topmargin      -60pt

%-----------------------------------------------------------
% Uncomment the following if you want to insert a watermark! 
%
%--> Watermark package settings: 
%\usepackage{draftwatermark}
%\SetWatermarkText{DRAFT}
%\SetWatermarkScale{0.5}
%\SetWatermarkColor[gray]{0.8}
%-------------------------------------------------

\begin{document} 
    
%-------------------------------------------------------------
%-- Make the header of the document                          |
%------>------>------>------>------>------>------>------>--> |
%--------------------------------------------------------------------------
%- The following produces the document header including the title.        |
%- The document header includes: the college/university name, faculty,    |
%  department, course number and title as well as the assignment/homework | 
%  title and due date.                                                    | 
%-------------------------------------------------------------------------|
%
\noindent % <-- need to have this first.
%
\begin{minipage}{.40\textwidth}
    {\color{darkred} \faSchool} { \textsc{\college}}{ } {\color{darkred} \faSchool}\\ 
    \small\textsc{Physics}
\end{minipage}%
\hfill	
\begin{minipage}{0.60\textwidth}%
    \raggedleft%
    {\Large \textsc{\coursenumber { } \coursetitle}\par}
    \doublerule % insert a double rule.
    \textsc{Teacher}: \instructor\\
\end{minipage}%
\vspace{2.8cm}
{
    \vspace{.3cm}
    \centering \large\myname \\
}
{
    \vspace{.2cm}
    \centering \large\rollnumber\\
}
{
    \vspace{.3cm}
    %--> Insert homework title and due date.
    \hrule\vspace{.2cm}
    \centering
    {\scshape 
        \Large \color{darkestblue}{\doctitle}{ }\textemdash{ }\small\bfseries\textsc{\semester}\par}
    \vspace{.3cm}    
}
{
    \hrule\vspace{.3cm}
    \centering  \small\duedate \\ 

}    

\vspace{3.5cm}



%
\tableofcontents

\clearpage
\section{Hyperfine Splitting}
\subsubsection*{Splitting of Na ground and excited states}
we know,
\begin{gather}
    \vec{\mathbf{F} } = \vec{\mathbf{I} }+\vec{\mathbf{J}}\\
    \Delta E = A \vec{\mathbf{I} }.\vec{\mathbf{J}}\\
    \implies \frac{A}{2}\left(\left\vert \vec{F} \right\vert^2 -\left\vert \vec{I} \right\vert^2 -\left\vert \vec{J}\right\vert^2\right)\\
     \implies \frac{A}{2}[F(F+1)-I(I+1)-J(J+1)]
\end{gather}
For each \(\vec{F}\) there is different levels in the range
\begin{gather}
    \left\vert I-J \right\vert \leq F\leq I+J
\end{gather}
If \(I\geq J\) there are 2J+1 levels and if \(J\geq I\) there are 2I+1 levels. \\
Ground state of Na(\(S _ \frac{1}{2}\) ) has \(I\geq J\)  and J = 1/2. Therefore, 2J+1 = 2 states. \\
Second excited state of Na (\(P_ \frac{3}{2}\) ) has \(I \geq J\)  and J = 3/2. Therefore, 2J+1 = 4 states.
\subsubsection*{Constant of proportionality A}
Ground state splitting,  
\begin{gather}
    h\Delta v = \frac{6.626X10^{-34}JsX 1772X10^{6}Xs^{-1}}{1.6X10^{-19}\frac{J}{eV}} = 7.33 \mu eV\\
    \Delta E(F=2) - \Delta E(F=1) = h\Delta v \\
    \implies \frac{A}{2}[2(3)-1(2)] = 7.33 \mu eV \\
    \implies 2A = 7.33 \mu eV\\
    \implies A = 3.67 \mu eV
\end{gather}
First excited state splitting,
\begin{gather}
    h\Delta v = \frac{6.626X10^{-34}JsX 192X10^{6}Xs^{-1}}{1.6X10^{-19}\frac{J}{eV}} = 0.795 \mu eV\\
    \Delta E(F=2) - \Delta E(F=1) = h\Delta v \\
    \implies \frac{A}{2}[2(3)-1(2)] = 0.795 \mu eV \\
    \implies 2A = 0.795 \mu eV\\
    \implies A = 0.397 \mu eV
\end{gather}
Second excited state splitting, between F=0 and F=1, 
\begin{gather}
    h\Delta v = \frac{6.626X10^{-34}JsX 17.1X10^{6}Xs^{-1}}{1.6X10^{-19}\frac{J}{eV}} = 0.07 \mu eV\\
    \Delta E(F=1) - \Delta E(F=0) = h\Delta v \\
    \implies \frac{A}{2}[1(2)-0(1)] = 0.07 \mu eV \\
    \implies A = 0.07 \mu eV
\end{gather}
Second excited state splitting, between F=1 and F=2,
\begin{gather}
    h\Delta v = \frac{6.626X10^{-34}JsX 36.6X10^{6}Xs^{-1}}{1.6X10^{-19}\frac{J}{eV}} = 0.152 \mu eV\\
    \Delta E(F=2) - \Delta E(F=1) = h\Delta v \\
    \implies \frac{A}{2}[2(3)-1(2)] = 0.152 \mu eV \\
    \implies 2A = 0.0152 \mu eV\\
    \implies A = 0.0757 \mu eV
\end{gather}
Second excited state splitting, between F=2 and F=3,
\begin{gather}
    h\Delta v = \frac{6.626X10^{-34}JsX 60.9X10^{6}Xs^{-1}}{1.6X10^{-19}\frac{J}{eV}} = 0.252 \mu eV\\
    \Delta E(F=3) - \Delta E(F=2) = h\Delta v \\
    \implies \frac{A}{2}[3(4)-2(3)] = 0.252 \mu eV \\
    \implies 3A = 0.252 \mu eV\\
    \implies A = 0.084 \mu eV
\end{gather}
\section{Deuteron}
\section{Cross-section and Beam Intensity for an interaction}      
\label{sec:1} 
\subsection*{ Cross-section for an interaction }
It is the measure of the probability of a collision taking place between particles. It is expressed as an area.\\
The total rate W $\propto$ NI \\
where, \\ N = number of exposed targets \\ I = flux of incoming particles per unit area per unit time. \\
Therefore, \(W = \sigma N I \)  \\
where \(\mathbf{\sigma}\) is \textbf{cross-section} and the \textbf{constant of proportionality} with dimensions of area.

\subsection*{Beam Intensity}
Assume a small length dx of solid perpendicular to the beam of area A. The change in intensity is -dI. This is equal to number of particles removed from the beam per unit area in length dx. 
Number of particles, N is 
\[
    N = n A dx \because \text{n is number of particles per unit volume }
\]
\begin{gather}
    W = \sigma_t I N = - A dI \\
    W = \sigma_t I (n A dx) = - A dI\\
    \implies \frac{dI}{I} = - n \sigma_t dx \\
    \implies I(x) = I_0 e^{-n\sigma_t x}
\end{gather}
\subsection*{Thickness of Lead required}
Given, 
\begin{equation}
    I(d) = \frac{I_{o} }{1000}
\end{equation}
\begin{equation}
    \rho(Pb) = 11300 kg m^{-3} 
\end{equation}
\begin{equation}
   m_{pb} (\text{mass of Pb}) = 207.21 u  
\end{equation}
\begin{equation}
    n = \frac{\rho}{m_{pb}}
\end{equation} 
from above 3 equations, 
\begin{equation}
    n = \frac{11300 kgm^{-3}}{207.21X1.66
    X10^{-27}kg} = 3.28 X 10^{28} m^{-3}  
\end{equation} 
\begin{equation}
    \sigma_t = 2.6X10^3 \text{barns} = 2.6X10^{-25} m^2
\end{equation}
\begin{equation}
    \frac{I(d)}{I_o} = e^{-n \sigma_t d}
\end{equation}
\begin{gather}
    10^{-3} = e^{-n \sigma_t d} \\
    3\ln (10) = 3.28X10^{28}X2.6X10^{-25}Xd \\
    d = 0.81 \text{mm} 
\end{gather}
\subsection*{Thickness of Aluminum required }
Given, 
\begin{equation}
    I(d) = \frac{I_{o} }{1000}
\end{equation}
\begin{equation}
    \rho(Al) = 2700 kg m^{-3} 
\end{equation}
\begin{equation}
   m_{Al} (\text{mass of Al}) = 26.29 u  
\end{equation}
\begin{equation}
    n = \frac{\rho}{m_{Al}}
\end{equation} 
from above 3 equations, 
\begin{equation}
    n = \frac{2700 kgm^{-3}}{26.29X1.66
    X10^{-27}kg} = 6.18 X 10^{28}  m^{-3}  
\end{equation} 
\begin{equation}
    \sigma_t = 13 \text{barns} = 13X10^{-28} m^2
\end{equation}
\begin{equation}
    \frac{I(d)}{I_o} = e^{-n \sigma_t d}
\end{equation}
\begin{gather}
    10^{-3} = e^{-n \sigma_t d} \\
    3\ln (10) = 6.18X10^{28}X13X10^{-28}Xd \\
    d = 85.98 \text{mm} 
\end{gather}
\section{Elastic scattering, De-excitation, Fission}
\subsection*{Attenuation rate}
Given,
\begin{gather}
    I_o = 10^5 s^{-1} \\
    \rho x = 10^{-1} kgm^{-2} \\
    \sigma_e = 20 mb\\
    \sigma_c = 70 b\\
    \sigma_f = 200 b \\
    A = 235\\
    N_A = \text{Avogadro's Number}
\end{gather}
\begin{gather}
    I(x) = I_o e^{-n \sigma_t x}\\
    n = \frac{\rho}{m_U} \\
    m_U = \frac{A}{N_A X 10^3} \text{ in Kg} = \frac{235}{6.023 X10 ^26} = 3.90X10^{-25} kg\\
    \sigma_t = \sigma_e + \sigma_c + \sigma_f = 270.02 X 10^{-28} m^2 \\
    nx = \frac{\rho x}{m_U} = \frac{10^{-1} }{3.90X10^{-25}}m^{-2} = 2.56X10^{23} m^{-2}\\
    n\sigma_{t}x = 6.92X10^{-3}\\
    e^{-n\sigma_{t}x} =  0.9931 
\end{gather}
\begin{equation}
    \boxed{\text{Attenuation rate,}\frac{I(x)}{I_{o}} = 0.9931}
\end{equation}
\subsection*{Number of fission reactions per second}
Replacing \(\sigma_t\) with \(\sigma_f\) to get only Intensity decreased by fission reaction.
\begin{gather}
    \frac{I_f(x)}{I_{o} } = e^{-n\sigma_{f}x}\\
    \implies nx = 2.56X10^{23}m^{-2} \text{ and } \sigma_f = 200 X 10^{-28}m^2 \\
    -n\sigma_{f}x = 5.12X10^{-3} \\
    e^{-n\sigma_{f}x }  = 0.9948\\
    \frac{I_f(x)}{I_{o} }= 0.9948
\end{gather}
Intensity decreased by fission is equal to \(I_{o} -I_{f} (x)\),
\begin{gather}
    I_{o} -I_{f} (x) = I_{o} (1-e^{-n\sigma_f}x ) = 10^5(1-0.9948)=510 s^{-1} 
\end{gather}
\begin{equation}
    \boxed{\text{Number of fission reactions per second =}510s^{-1} }
\end{equation}
\subsubsection*{Elastic scattering}
Attenuation due to elastic scattering,
\begin{gather}
    \frac{I_e(x)}{I_{o} } = e^{-n\sigma_{e}x}\\
    \implies nx = 2.56X10^{23}m^{-2} \text{ and } \sigma_e = 20 X 10^{-31}m^2 \\
    -n\sigma_{e}x = 5.12X10^{-7} \\
    e^{-n\sigma_{e}x }  = 0.999999488\\
    \frac{I_e(x)}{I_{o} }=0.999999488
\end{gather}
Number of scattered per second = \(I_{o} (1-\frac{I_e(x)}{I_{o} }) = 5.1X10^{-2} s^{-1} \). \\
Area of sphere at 10m = \(400 \pi m^2\) \\
\begin{equation}
    \boxed{\text{Flux = } \frac{I}{A} = \frac{5.1X10^{-2}}{400 \pi}m^{-2}s^{-1}  = 4.074X10^{-5}m^{-2}s^{-1} }
\end{equation}
For gamma rays,
\begin{gather}
    \frac{I_c(x)}{I_{o} } = e^{-n\sigma_{c}x}\\
    \implies nx = 2.56X10^{23}m^{-2} \text{ and } \sigma_c = 70 X 10^{-28}m^2 \\
    -n\sigma_{c}x = 1.792X10^{-3} \\
    e^{-n\sigma_{c}x }  = 0.9982\\
    \frac{I_{c} (x)}{I_{o} }=0.9982
\end{gather}
Number of scattered per second = \(I_{o} (1-\frac{I_c(x)}{I_{o} }) = 179 s^{-1} \). \\
Area of sphere at 10m = \(400 \pi m^2\) \\
\begin{equation}
    \boxed{\text{Flux = } \frac{I}{A} = \frac{179}{400 \pi}m^{-2}s^{-1}  = 0.1424m^{-2}s^{-1} }
\end{equation}
\end{document} 
