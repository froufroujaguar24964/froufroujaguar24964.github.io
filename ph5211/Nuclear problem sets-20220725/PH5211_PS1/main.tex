% !TeX spellcheck = en_US
\documentclass[11pt, a4paper]{article}

% Set the title of the current document to be produced.
\newcommand{\doctitle}{Assignment I}
% Command for the due date of the homework.
\newcommand{\duedate}{\color{rltblue}{\faCalendarCheckO { }Due date: August 10th, in class \faCalendarCheckO	}}
\newcommand{\myname}{\color{rltred} {}\textbf{Chaganti}  \textbf{Kamaraja}  \textbf{Siddhartha} }
\newcommand{\rollnumber}{\color{rltred}{}\textbf{EP20B012} }
%------------------------------------------------------------
% Import commands for both teacher and course information.  | 
% NOTE: Change your teacher and course info in these files. |
%------>------>------>------>------>------>------>------>-->|
%-------------------------------------------------
% Teacher-specific commands                      |
%---------------                                 |
%-> Instructions: change your teacher info here. |
%------->------>------>------>------>------>---->|
%
\newcommand{\instructor}{Vaibhav Madhok}
\newcommand{\office}{HSB 234-2}
\newcommand{\hours}{By appointment}
\newcommand{\phone}{044 - 2257}
\newcommand{\college}{IIT Madras}
\newcommand{\email}{madhok@iitm.ac.in}
\newcommand{\faculty}{Assistant Professor}
\newcommand{\department}{Physics}
                              %|
%-------------------------------------------------
% Course-specific commands                       |
%---------------                                 |
%-> Instructions: change your course info here.  |
%------->------>------>------>------>------>---->|
%
\newcommand{\semester}{July-Nov 2022}
\newcommand{\csection}{00001 \& 00002}
\newcommand{\ponderation}{2-4-3 (Theory-Lab-Homework)}
\newcommand{\coursetitle}{High Energy Physics}
\newcommand{\coursenumber}{PH5211}
\newcommand{\prerequisite}{All porgram courses semesters 1-4}
                               %|   
%
%------------------------------------------------------------
%-- Import packages and custom command definitons.          |
%------>------>------>------>------>------>------>------>-->|
\input{includes/packages}                                  %|  
\input{includes/custom-commands}   
%
%---> Genereate & inject metadata describing                |
%     the produced document                                 |
\input{includes/metadata}                                  %|
%------------------------------------------------------------

\topmargin      -60pt

%-----------------------------------------------------------
% Uncomment the following if you want to insert a watermark! 
%
%--> Watermark package settings: 
%\usepackage{draftwatermark}
%\SetWatermarkText{DRAFT}
%\SetWatermarkScale{0.5}
%\SetWatermarkColor[gray]{0.8}
%-------------------------------------------------

\begin{document} 
    
%-------------------------------------------------------------
%-- Make the header of the document                          |
%------>------>------>------>------>------>------>------>--> |
\input{includes/document-header}


%
\tableofcontents

\clearpage
    
\section{Problem 1}      
\label{sec:1} 
Fermi's golden rule:- The rate at which the scattering occurs will be proportional to the matrix element squared \(\left\vert V_{if}^\prime  \right\vert^2 \) 
\[
    V_{if} ^\prime  = \int \psi^*_{f} V(r) \psi_{i} d\tau
\]
Given,
\[
    \psi_{f} = e^ {(i \bar{k_f}.\bar{r})}
\]
\[
    \psi_{i} = e^ {(i \bar{k_i}.\bar{r}) }
\]
Potential V(r) is,
\[
    V(r) = -\frac{Ze^2}{4 \pi \epsilon_{o}}\int \frac{\rho_e(r^\prime )}{\left\vert \bar{r}-\bar{r^\prime } \right\vert }d\tau^\prime  
\]
\[
    V_{if} ^\prime = \int   (e^ {(-i \bar{k_f}.\bar{r})})(e^ {(i \bar{k_i}.\bar{r}) })\left(-\frac{Ze^2}{4 \pi \epsilon_{o}}\int \frac{\rho_e(r^\prime )}{\left\vert \bar{r}-\bar{r^\prime } \right\vert }d\tau^\prime\right) d\tau
\]
\[
    V_{if}^\prime = -\frac{Ze^2}{4 \pi \epsilon_{o}} \int e^{i\bar{q}.\bar{r}}\left(\int\frac{\rho_e(r^\prime )}{\left\vert \bar{r}-\bar{r^\prime } \right\vert }d\tau^\prime\right)d\tau \hspace{.5cm} (\because \bar{q} = \bar{k_i}-\bar{k_{f}})
\]
multiplying with \(e^{+i\bar{q}.\bar{r^\prime }}\) and \(e^{-i\bar{q}.\bar{r^\prime }}\) and writing \(\bar{r}-\bar{r^\prime } = \bar{R}\) also, \(\bar{R} = \bar{r}-\bar{r^\prime }  \to  dR = dr\) 
\[
    V_{if}^\prime = -\frac{Ze^2}{4 \pi \epsilon_{o}} \int e^{i\bar{q}.\bar{R}}\left(\int\frac{\rho_e(r^\prime )e^{i \bar{q}.\bar{r^\prime }}}{\left\vert \bar{R} \right\vert }d\tau^\prime\right)d\tau \hspace{.5cm} 
\]
Form Factor, \(F(\bar{q}) = \int \rho_e(r^\prime)e^{i \bar{q}.\bar{r^\prime }}d\tau^\prime\) 
\[
    V_{if}^\prime = -\frac{Ze^2}{4 \pi \epsilon_{o}} \int \frac{e^{i\bar{q}.\bar{R}}}{\left\vert \bar{R} \right\vert}F(\bar{q})d\tau \hspace{.5cm} 
\]



\section{Homework 1.2}
\label{sec:1.2}


\section{Homework 1.3}
\label{sec:1.3}


    
\end{document} 