<<<<<<< HEAD
\lecture{5}{\today}{Continution}
\textbf{Measure spin: Hyper fine splitting}\\
Perturbation $\alpha$ \(\vec{\mu}.\bar{B}_{e^- in atom}\) $\alpha$ \(\vec{I}.\vec{J}\) where, \(\vec{J}\) = total angular momentum of $e^- $state
\begin{definition}
    \[
        \vec{F} = \vec{I}+ \vec{J} \implies 
        \left\vert \vec{I}- \vec{J} \right\vert \leq F \leq \left\vert \vec{I}+ \vec{J} \right\vert 
    \]
\end{definition}

H in 1s \(\implies \vec{J} = \frac{1}{2}\) and \(\vec{I} = \frac{1}{2}\)
So, 1s splits into 2 states F=1 and F=0. with energy difference is $5.9 X 10^{-6} eV. $\\
Gamma wave is released when transition happens with \(\nu = 1.42 GHz\) or \(\lambda = 21cm\). The half life time of the state F=1 is \(\tau = 10^7 years\). \\
\textbf{to measure consider}  
\[
    \left\vert \vec{F} \right\vert^2 = \left\vert \vec{I} \right\vert^2 + \left\vert \vec{J} \right\vert^2 + 2\vec{I}.\vec{J}
\]
\[
    \vec{I}.\vec{J} = \frac{1}{2} (\left\vert \vec{F} \right\vert^2 -\left\vert \vec{I} \right\vert^2 - \left\vert \vec{J} \right\vert^2 )
\]
\[
    \vec{I}.\vec{J} = \frac{1}{2}(F(F+1)-I(I+1)-J(J+1))
\]
\[

\]
=======
\subsection{Magnetic moments}
\[
    \bar{\mu} = \frac{e \overline{h} }{2 m_{p} } \bar{l}
\]
\[
    n: \bar{\mu} = 0
\]
\[
    p: \bar{\mu} = \mu_n \bar{l}
\]
where $\mu_n$ = nuclear magneton\adfj
\textbf{Intrinsic}  \textbf{spin} 
\[
    \bar{\mu} = g_{s} \bar{s} \mu_{N} 
\]
However
\[
    proton \implies  g_s = 5.59
\]
\[
    neutron \implies g_s = -3.82
\]
>>>>>>> feature_in_progress
