\lecture{2}{26 July 2022}{Second Lecture}
\section{Thermodynamic Laws}
\subsection{Zeroth Law }
\begin{definition}
    If A is in Thermodynamic equilibrium with B and B is in Thermodynamic equilibrium with C, then C is in thermodynamic equilibrium with A. This law defines the notion of temperature (T).
    \begin{figure}[H]
        \centering
        \incfig{zeroth_law}
        \caption{Zeroth Law}
        \label{fig:zeroth_law}
    \end{figure}
\end{definition}
\subsection{First Law}
\textbf{Work Done}\\
Conjugate variables\\
P \(\to\) \(\Delta V\) \\
\(\mu\)(Chemical Potential) \(\to \) \(\Delta N\)  \\
F \(\to \) \(\Delta L\)  \\
\textbf{Quasi-static process}: \( PdV \) = infinitesimal work done.
\[
    \Delta W = \int _{A}^{B} P dV
\]
Since, Work done is path dependent it is inexact differential and written as \( d(bar)W\)
\textbf{Heat exchange} : It is also path dependent so, an inexact differential \(d(bar)Q = SdT\) 
\begin{definition}[First Law]
    There exists a quantity called "Internal Energy" 
    \[
        \Delta U = \Delta Q - \Delta W
    \]
    \[
        dU = dQ - dW
    \]
    dU is an exact differential, depends only on initial and final states. \\
    It is an extensive quantity.\\
\end{definition}
\textbf{Ideal gas}: State is defined by (P, V, T) \\
Equation of state is given by \(P V = N k_{b} T\). This is a constraint so, independent variables reduce to 2.
\[
    U(P,V)\implies 
    dU = \left(\frac{\partial U}{\partial P}\right)_{V} dP + \left(\frac{\partial U}{\partial V}\right)_{P} dV
\]
\[
    U(T,V)\implies 
    dU = \left(\frac{\partial U}{\partial T}\right)_{V} dT + \left(\frac{\partial U}{\partial V}\right)_{T} dV
\]
\[
    U(P,T)\implies 
    dU = \left(\frac{\partial U}{\partial P}\right)_{T} dP + \left(\frac{\partial U}{\partial T}\right)_{P} dT
\]
\begin{definition}[Specific Heat]
   \(dQ = C_{V} dT\) or \(C_{P} dT\)  
   response of the system to heat.
\end{definition}
\textbf{Constant Volume Process}: 
\[
    dU = dQ - PdV
\]
we know for a constant volume process \(dQ = C_{V} dT\),
\[
    dU = C_{V} dT - PdV
\]
we also know, 
\[
    dU = \left(\frac{\partial U}{\partial T}\right)_{V} dT + \left(\frac{\partial U}{\partial V}\right)_{T} dV
\]
\(\because \) V is constant \(\implies \) dV = 0.\\

Equating the above 2 equations gives, 
\[
    C_{V} = \left(\frac{\partial U}{\partial T}\right)_{V}
\]
\textbf{Constant}  \textbf{Pressure} \textbf{process} :\\

we know, 
\[
    dQ = C_{P} dT
\]
\[
    dU = C_{P} dT - PdV
\]
we also know,
\[
    dU = \left(\frac{\partial U}{\partial P}\right)_{T} dP + \left(\frac{\partial U}{\partial T}\right)_{P} dT
\] 
dV term is there in the equation. Since, V is a function of P and T we can write,
\[
    dV = \left(\frac{\partial V}{\partial P}\right)_{T} dP + \left(\frac{\partial V}{\partial T}\right)_{P} dT
\]
Substituting dV in first equation,
\[
    dU = C_{P} dT - P\left(\left(\frac{\partial V}{\partial P}\right)_{T} dP + \left(\frac{\partial V}{\partial T}\right)_{P} dT\right)
\]
since, constant pressure dP = 0. Therefore, 
\[
    C_{P} = P \left(\frac{\partial V}{\partial T}\right)_{P} +  \left(\frac{\partial U}{\partial T}\right)_{P}
\]
\begin{definition}[Enthalpy, H]
    Thermodynamic Potential, Enthalpy H = U + PV
    \[
        dH = dU + PdV
    \]
    \[
        C_{P} = \left(\frac{\partial H}{\partial T}\right)_{P} 
    \]
\end{definition}
\textbf{Heat} \textbf{Engine} :\\
\textbf{Isothermal} \textbf{Process} \\
\begin{figure}[H]
    \centering
    \incfig{cyclic}
    \caption{cyclic process}
    \label{fig:cyclic}
\end{figure}
\textbf{Adiabatic} \textbf{Process} : Constant heat Process.
\[
    dQ = 0
\]
\[
    dU = -dW
\]
\[
    dU = -PdV
\]
\begin{eg}[Ideal Gas Adibatic free expansion]
    \begin{figure}[H]
        \centering
        \incfig{expansion}
        \caption{Free expansion}
        \label{fig:expansion}
    \end{figure}
    \[
        dQ = 0 \because Adiabatic
    \]
    \[
        dW = 0 \because Free Expansion
    \]Since, Free expansion. 
    \[
        \therefore dU = 0 \to f(P,V) = 0
    \]
\end{eg}
