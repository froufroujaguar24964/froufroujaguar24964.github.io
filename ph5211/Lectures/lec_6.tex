\lecture{6}{\today}{duetron}
This is the anologue of atomic H for nucleus \(\because ^2_1 H \equiv D\equiv d\).\\
Mass from a spectrometer using the tricks we discussed(see krane)\\
\(m_D = 2.014101771(15) u\) \(\implies  B_D = 2.22 eV \ll 16eV\)\\
V small B \(\implies \) no excited states \\
\textbf{No \(\gamma\) spectroscopy, however, we can learn about V(r) } 
%%\begin{figure}[H]
   %% \centering
    %%\incfig{las}
    %%\caption{title}
  %%  \label{fig:las}
%%\end{figure}
\[
    -\frac{\hbar^2 }{2m} \frac{d^{2}u }{dr^2} + V(r)u = E u
\] where u = \(\frac{\Psi(r)}{r}\) \\
\textbf{Inside Well  } 
\[u_i = A \sin k_{1}r + B \cos  k_1 r \]
\[u_o = C e^{-k_2 r} + D e^{k2 r}\] \\
with,
\[
    k_1 = \sqrt{\frac{2m(E+V_0)}{\hbar^2}} ,  
    k_2 = \sqrt{\frac{-2mE}{\hbar^2}} 
\]

\begin{enumerate}
    \item finite at r \(\to  0 \) \(\implies B=0\) 
    \item \(\Psi \to 0\) at \(r\to \infty \)  \(\implies  D = 0\) 
    \item \(\frac{du_i}{dr}_R = \frac{du_o}{dr}_R\) \(\implies Ak_1\cos k_1 R = -C k_2 e^{-k_2R} \)  
    \item $u_i(R) = u_o(R)$ \(\implies  A \sin k_1 R = C e^{-k_2 R}\) 
\end{enumerate}
from 3 and 4 \(\implies \) \( k_1 \cot k_1 R = -k_2\) depends on $V_o$ , R \(\implies V_{o} = 35 MeV\) 
%% 2Figure

\[
    p + p \to D + e^+ +v_e + 0.42 MeV
\]
\[
    p + D \to ^{3}He + \gamma + 5.5 MeV
\]
\[
    ^3He + ^3He \to ^4He + p + p + \gamma + 12.98 MeV
\]
Total
\[
    4p \to \ce{^4He} + 2 e^+ + 2 v_e + 24.8 MeV
\]
The $ 24.8MeV$ is sunshine.