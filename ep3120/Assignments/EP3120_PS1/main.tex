% !TeX spellcheck = en_US
\documentclass[11pt, a4paper]{article}

% Set the title of the current document to be produced.
\newcommand{\doctitle}{Assignment II}
% Command for the due date of the homework.
\newcommand{\duedate}{\color{rltblue}{\faCalendarCheckO { }Due date: August 16th, in class \faCalendarCheckO	}}
\newcommand{\myname}{\color{rltred} {}\textbf{Chaganti}  \textbf{Kamaraja}  \textbf{Siddhartha} }
\newcommand{\rollnumber}{\color{rltred}{}\textbf{EP20B012} }
%------------------------------------------------------------
% Import commands for both teacher and course information.  | 
% NOTE: Change your teacher and course info in these files. |
%------>------>------>------>------>------>------>------>-->|
%-------------------------------------------------
% Teacher-specific commands                      |
%---------------                                 |
%-> Instructions: change your teacher info here. |
%------->------>------>------>------>------>---->|
%
\newcommand{\instructor}{Vaibhav Madhok}
\newcommand{\office}{HSB 234-2}
\newcommand{\hours}{By appointment}
\newcommand{\phone}{044 - 2257}
\newcommand{\college}{IIT Madras}
\newcommand{\email}{madhok@iitm.ac.in}
\newcommand{\faculty}{Assistant Professor}
\newcommand{\department}{Physics}
                              %|
%-------------------------------------------------
% Course-specific commands                       |
%---------------                                 |
%-> Instructions: change your course info here.  |
%------->------>------>------>------>------>---->|
%
\newcommand{\semester}{July-Nov 2022}
\newcommand{\csection}{00001 \& 00002}
\newcommand{\ponderation}{2-4-3 (Theory-Lab-Homework)}
\newcommand{\coursetitle}{High Energy Physics}
\newcommand{\coursenumber}{PH5211}
\newcommand{\prerequisite}{All porgram courses semesters 1-4}
                               %|   
%
%------------------------------------------------------------
%-- Import packages and custom command definitions.          |
%------>------>------>------>------>------>------>------>-->|
\input{includes/packages}                                  %|  
\input{includes/custom-commands}   
%
%---> Generate & inject metadata describing                |
%     the produced document                                 |
\input{includes/metadata}                                  %|
%------------------------------------------------------------

\topmargin      -60pt

%-----------------------------------------------------------
% Uncomment the following if you want to insert a watermark! 
%
%--> Watermark package settings: 
%\usepackage{draftwatermark}
%\SetWatermarkText{DRAFT}
%\SetWatermarkScale{0.5}
%\SetWatermarkColor[gray]{0.8}
%-------------------------------------------------

\begin{document} 
    
%-------------------------------------------------------------
%-- Make the header of the document                          |
%------>------>------>------>------>------>------>------>--> |
\input{includes/document-header}


%
\tableofcontents

\clearpage
\section{Hyperfine Splitting}
\subsubsection*{Splitting of Na ground and excited states}
we know,
\begin{gather}
    \vec{\mathbf{F} } = \vec{\mathbf{I} }+\vec{\mathbf{J}}\\
    \Delta E = A \vec{\mathbf{I} }.\vec{\mathbf{J}}\\
    \implies \frac{A}{2}\left(\left\vert \vec{F} \right\vert^2 -\left\vert \vec{I} \right\vert^2 -\left\vert \vec{J}\right\vert^2\right)\\
     \implies \frac{A}{2}[F(F+1)-I(I+1)-J(J+1)]
\end{gather}
For each \(\vec{F}\) there is different levels in the range
\begin{gather}
    \left\vert I-J \right\vert \leq F\leq I+J
\end{gather}
If \(I\geq J\) there are 2J+1 levels and if \(J\geq I\) there are 2I+1 levels. \\
Ground state of Na(\(S _ \frac{1}{2}\) ) has \(I\geq J\)  and J = 1/2. Therefore, 2J+1 = 2 states. \\
Second excited state of Na (\(P_ \frac{3}{2}\) ) has \(I \geq J\)  and J = 3/2. Therefore, 2J+1 = 4 states.
\subsubsection*{Constant of proportionality A}
Ground state splitting,  
\begin{gather}
    h\Delta v = \frac{6.626X10^{-34}JsX 1772X10^{6}Xs^{-1}}{1.6X10^{-19}\frac{J}{eV}} = 7.33 \mu eV\\
    \Delta E(F=2) - \Delta E(F=1) = h\Delta v \\
    \implies \frac{A}{2}[2(3)-1(2)] = 7.33 \mu eV \\
    \implies 2A = 7.33 \mu eV\\
    \implies A = 3.67 \mu eV
\end{gather}
First excited state splitting,
\begin{gather}
    h\Delta v = \frac{6.626X10^{-34}JsX 192X10^{6}Xs^{-1}}{1.6X10^{-19}\frac{J}{eV}} = 0.795 \mu eV\\
    \Delta E(F=2) - \Delta E(F=1) = h\Delta v \\
    \implies \frac{A}{2}[2(3)-1(2)] = 0.795 \mu eV \\
    \implies 2A = 0.795 \mu eV\\
    \implies A = 0.397 \mu eV
\end{gather}
Second excited state splitting, between F=0 and F=1, 
\begin{gather}
    h\Delta v = \frac{6.626X10^{-34}JsX 17.1X10^{6}Xs^{-1}}{1.6X10^{-19}\frac{J}{eV}} = 0.07 \mu eV\\
    \Delta E(F=1) - \Delta E(F=0) = h\Delta v \\
    \implies \frac{A}{2}[1(2)-0(1)] = 0.07 \mu eV \\
    \implies A = 0.07 \mu eV
\end{gather}
Second excited state splitting, between F=1 and F=2,
\begin{gather}
    h\Delta v = \frac{6.626X10^{-34}JsX 36.6X10^{6}Xs^{-1}}{1.6X10^{-19}\frac{J}{eV}} = 0.152 \mu eV\\
    \Delta E(F=2) - \Delta E(F=1) = h\Delta v \\
    \implies \frac{A}{2}[2(3)-1(2)] = 0.152 \mu eV \\
    \implies 2A = 0.0152 \mu eV\\
    \implies A = 0.0757 \mu eV
\end{gather}
Second excited state splitting, between F=2 and F=3,
\begin{gather}
    h\Delta v = \frac{6.626X10^{-34}JsX 60.9X10^{6}Xs^{-1}}{1.6X10^{-19}\frac{J}{eV}} = 0.252 \mu eV\\
    \Delta E(F=3) - \Delta E(F=2) = h\Delta v \\
    \implies \frac{A}{2}[3(4)-2(3)] = 0.252 \mu eV \\
    \implies 3A = 0.252 \mu eV\\
    \implies A = 0.084 \mu eV
\end{gather}
\section{Deuteron}
\section{Cross-section and Beam Intensity for an interaction}      
\label{sec:1} 
\subsection*{ Cross-section for an interaction }
It is the measure of the probability of a collision taking place between particles. It is expressed as an area.\\
The total rate W $\propto$ NI \\
where, \\ N = number of exposed targets \\ I = flux of incoming particles per unit area per unit time. \\
Therefore, \(W = \sigma N I \)  \\
where \(\mathbf{\sigma}\) is \textbf{cross-section} and the \textbf{constant of proportionality} with dimensions of area.

\subsection*{Beam Intensity}
Assume a small length dx of solid perpendicular to the beam of area A. The change in intensity is -dI. This is equal to number of particles removed from the beam per unit area in length dx. 
Number of particles, N is 
\[
    N = n A dx \because \text{n is number of particles per unit volume }
\]
\begin{gather}
    W = \sigma_t I N = - A dI \\
    W = \sigma_t I (n A dx) = - A dI\\
    \implies \frac{dI}{I} = - n \sigma_t dx \\
    \implies I(x) = I_0 e^{-n\sigma_t x}
\end{gather}
\subsection*{Thickness of Lead required}
Given, 
\begin{equation}
    I(d) = \frac{I_{o} }{1000}
\end{equation}
\begin{equation}
    \rho(Pb) = 11300 kg m^{-3} 
\end{equation}
\begin{equation}
   m_{pb} (\text{mass of Pb}) = 207.21 u  
\end{equation}
\begin{equation}
    n = \frac{\rho}{m_{pb}}
\end{equation} 
from above 3 equations, 
\begin{equation}
    n = \frac{11300 kgm^{-3}}{207.21X1.66
    X10^{-27}kg} = 3.28 X 10^{28} m^{-3}  
\end{equation} 
\begin{equation}
    \sigma_t = 2.6X10^3 \text{barns} = 2.6X10^{-25} m^2
\end{equation}
\begin{equation}
    \frac{I(d)}{I_o} = e^{-n \sigma_t d}
\end{equation}
\begin{gather}
    10^{-3} = e^{-n \sigma_t d} \\
    3\ln (10) = 3.28X10^{28}X2.6X10^{-25}Xd \\
    d = 0.81 \text{mm} 
\end{gather}
\subsection*{Thickness of Aluminum required }
Given, 
\begin{equation}
    I(d) = \frac{I_{o} }{1000}
\end{equation}
\begin{equation}
    \rho(Al) = 2700 kg m^{-3} 
\end{equation}
\begin{equation}
   m_{Al} (\text{mass of Al}) = 26.29 u  
\end{equation}
\begin{equation}
    n = \frac{\rho}{m_{Al}}
\end{equation} 
from above 3 equations, 
\begin{equation}
    n = \frac{2700 kgm^{-3}}{26.29X1.66
    X10^{-27}kg} = 6.18 X 10^{28}  m^{-3}  
\end{equation} 
\begin{equation}
    \sigma_t = 13 \text{barns} = 13X10^{-28} m^2
\end{equation}
\begin{equation}
    \frac{I(d)}{I_o} = e^{-n \sigma_t d}
\end{equation}
\begin{gather}
    10^{-3} = e^{-n \sigma_t d} \\
    3\ln (10) = 6.18X10^{28}X13X10^{-28}Xd \\
    d = 85.98 \text{mm} 
\end{gather}
\section{Elastic scattering, De-excitation, Fission}
\subsection*{Attenuation rate}
Given,
\begin{gather}
    I_o = 10^5 s^{-1} \\
    \rho x = 10^{-1} kgm^{-2} \\
    \sigma_e = 20 mb\\
    \sigma_c = 70 b\\
    \sigma_f = 200 b \\
    A = 235\\
    N_A = \text{Avogadro's Number}
\end{gather}
\begin{gather}
    I(x) = I_o e^{-n \sigma_t x}\\
    n = \frac{\rho}{m_U} \\
    m_U = \frac{A}{N_A X 10^3} \text{ in Kg} = \frac{235}{6.023 X10 ^26} = 3.90X10^{-25} kg\\
    \sigma_t = \sigma_e + \sigma_c + \sigma_f = 270.02 X 10^{-28} m^2 \\
    nx = \frac{\rho x}{m_U} = \frac{10^{-1} }{3.90X10^{-25}}m^{-2} = 2.56X10^{23} m^{-2}\\
    n\sigma_{t}x = 6.92X10^{-3}\\
    e^{-n\sigma_{t}x} =  0.9931 
\end{gather}
\begin{equation}
    \boxed{\text{Attenuation rate,}\frac{I(x)}{I_{o}} = 0.9931}
\end{equation}
\subsection*{Number of fission reactions per second}
Replacing \(\sigma_t\) with \(\sigma_f\) to get only Intensity decreased by fission reaction.
\begin{gather}
    \frac{I_f(x)}{I_{o} } = e^{-n\sigma_{f}x}\\
    \implies nx = 2.56X10^{23}m^{-2} \text{ and } \sigma_f = 200 X 10^{-28}m^2 \\
    -n\sigma_{f}x = 5.12X10^{-3} \\
    e^{-n\sigma_{f}x }  = 0.9948\\
    \frac{I_f(x)}{I_{o} }= 0.9948
\end{gather}
Intensity decreased by fission is equal to \(I_{o} -I_{f} (x)\),
\begin{gather}
    I_{o} -I_{f} (x) = I_{o} (1-e^{-n\sigma_f}x ) = 10^5(1-0.9948)=510 s^{-1} 
\end{gather}
\begin{equation}
    \boxed{\text{Number of fission reactions per second =}510s^{-1} }
\end{equation}
\subsubsection*{Elastic scattering}
Attenuation due to elastic scattering,
\begin{gather}
    \frac{I_e(x)}{I_{o} } = e^{-n\sigma_{e}x}\\
    \implies nx = 2.56X10^{23}m^{-2} \text{ and } \sigma_e = 20 X 10^{-31}m^2 \\
    -n\sigma_{e}x = 5.12X10^{-7} \\
    e^{-n\sigma_{e}x }  = 0.999999488\\
    \frac{I_e(x)}{I_{o} }=0.999999488
\end{gather}
Number of scattered per second = \(I_{o} (1-\frac{I_e(x)}{I_{o} }) = 5.1X10^{-2} s^{-1} \). \\
Area of sphere at 10m = \(400 \pi m^2\) \\
\begin{equation}
    \boxed{\text{Flux = } \frac{I}{A} = \frac{5.1X10^{-2}}{400 \pi}m^{-2}s^{-1}  = 4.074X10^{-5}m^{-2}s^{-1} }
\end{equation}
For gamma rays,
\begin{gather}
    \frac{I_c(x)}{I_{o} } = e^{-n\sigma_{c}x}\\
    \implies nx = 2.56X10^{23}m^{-2} \text{ and } \sigma_c = 70 X 10^{-28}m^2 \\
    -n\sigma_{c}x = 1.792X10^{-3} \\
    e^{-n\sigma_{c}x }  = 0.9982\\
    \frac{I_{c} (x)}{I_{o} }=0.9982
\end{gather}
Number of scattered per second = \(I_{o} (1-\frac{I_c(x)}{I_{o} }) = 179 s^{-1} \). \\
Area of sphere at 10m = \(400 \pi m^2\) \\
\begin{equation}
    \boxed{\text{Flux = } \frac{I}{A} = \frac{179}{400 \pi}m^{-2}s^{-1}  = 0.1424m^{-2}s^{-1} }
\end{equation}
\end{document} 
