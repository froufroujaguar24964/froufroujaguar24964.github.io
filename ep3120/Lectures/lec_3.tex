\lecture{3}{27 July 2022}{Third Lecture}
\[
    dU = dQ - dW
\]
\textbf{Free} \textbf{expansion} 
Irreversible Process
\[
    \Delta W = 0, \Delta Q = 0 \implies \Delta U = 0
\]
\[
    U(P,V,T) = U(V,T) \left(\because PV = Nk_{B}T\right)
\]
\[
    \implies U(V_{i} , T) = U(V_{f} , T)
\]
Internal energy is purely a function of T.
\[
    C_{V} =\left(\frac{\partial U}{\partial T}\right)_{V}  = \frac{dU}{dT}
\]
\[
    H = U + PV = U(T)+N k_{B} T 
\]
\[
    \implies H=H(T) 
\]
\[
   \because C_{P} = \left(\frac{\partial H}{\partial T}\right)_{P} 
\]
\[
   \implies C_P = \frac{dU}{dT} + N k_{B} 
\]
\[
    \implies C_P = C_V + N k_{B} \implies \gamma = \frac{C_{P} }{C_{V}}>1
\]\newpage
\textbf{Adiabatic Process [Ideal Gas]}:\\
\[
    dQ = 0
\]
\[
    \implies dU = dW = -PdV
\]
\[
    C_{V} dT = -P dV
\]
\[
    C_{V} PdV + C_{V} VdP + P N k_{B} dV = 0 \left(\because PdV + VdP = N k_{B} dT \right)
\]
\[
  \implies   C_{P} P dV + C_{V} V dP = 0
\]
\[
    \implies \gamma \frac{dV}{V} + \frac{dP}{P} = 0
\]
\[
    \implies PV^\gamma = Constant.
\]
\begin{note}
    In class a question is raised which is worth discussing.
    \begin{exercise}
        Ideal gas law \(\implies PV = \) constant for free expansion(Adiabatic Irreversible process.)
        Adiabatic Process \(\implies PV^\gamma =\)  constant.
        \[
            P_1 V_1 = P_2 V_2
        \]
        \[
            P_1 V_1 ^ \gamma = P_2 V_2 ^ \gamma
        \]
        dividing both equations gives\dots
        \[
            V_1 ^ {\gamma-1} = V_2 ^ {\gamma -1} \implies V_1 = V_2 
        \]
        But the process is free \textbf{expansion}.
        Can you explain why?
        \begin{answer}
            We obtained \(PV^\gamma =\) constant equation by taking $dW = -PdV$ which means the process is reversible.\\
            Where as the free expansion process is Irreversible so, we cannot use \(PV^\gamma =\) constant for this process.
        \end{answer}
    \end{exercise}
\end{note}
\subsection{Second Law}
\begin{definition}[Second Law]
    \begin{enumerate}
        \item The sole amount of a cyclic process cannot be the conversion of heat to work. 
        \item (corollary) Efficiency of a carnot engine \(\neq  1\) . 
        \item It is not possible to have a cyclic process whose sole outcome is transfer of heat form cooler bath to a heater bath.
    \end{enumerate}
\end{definition}
\begin{figure}[H]
    \centering
    \incfig{carnot}
    \caption{carnot cycle}
    \label{fig:carnot}
\end{figure}
