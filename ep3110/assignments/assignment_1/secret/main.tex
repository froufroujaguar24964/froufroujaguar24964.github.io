\documentclass[11pt, a4paper]{article}

% Set the title of the current document to be produced.
\newcommand{\doctitle}{Assignment I}
% Command for the due date of the homework.
\newcommand{\duedate}{\color{rltblue}{\faCalendarCheckO { }Due date: August 11th, before midnight \faCalendarCheckO	}}
\newcommand{\myname}{\color{rltred} {}\textbf{Chaganti}  \textbf{Kamaraja}  \textbf{Siddhartha} }
\newcommand{\rollnumber}{\color{rltred}{}\textbf{EP20B012} }
%------------------------------------------------------------
% Import commands for both teacher and course information.  | 
% NOTE: Change your teacher and course info in these files. |
%------>------>------>------>------>------>------>------>-->|
%-------------------------------------------------
% Teacher-specific commands                      |
%---------------                                 |
%-> Instructions: change your teacher info here. |
%------->------>------>------>------>------>---->|
%
\newcommand{\instructor}{Vaibhav Madhok}
\newcommand{\office}{HSB 234-2}
\newcommand{\hours}{By appointment}
\newcommand{\phone}{044 - 2257}
\newcommand{\college}{IIT Madras}
\newcommand{\email}{madhok@iitm.ac.in}
\newcommand{\faculty}{Assistant Professor}
\newcommand{\department}{Physics}
                              %|
%-------------------------------------------------
% Course-specific commands                       |
%---------------                                 |
%-> Instructions: change your course info here.  |
%------->------>------>------>------>------>---->|
%
\newcommand{\semester}{July-Nov 2022}
\newcommand{\csection}{00001 \& 00002}
\newcommand{\ponderation}{2-4-3 (Theory-Lab-Homework)}
\newcommand{\coursetitle}{High Energy Physics}
\newcommand{\coursenumber}{PH5211}
\newcommand{\prerequisite}{All porgram courses semesters 1-4}
                               %|   
%
%------------------------------------------------------------
%-- Import packages and custom command definitons.          |
%------>------>------>------>------>------>------>------>-->|
\input{includes/packages}                                  %|  
\input{includes/custom-commands}   
%
%---> Genereate & inject metadata describing                |
%     the produced document                                 |
\input{includes/metadata}                                  %|
%------------------------------------------------------------

\topmargin      -60pt

%-----------------------------------------------------------
% Uncomment the following if you want to insert a watermark! 
%
%--> Watermark package settings: 
%\usepackage{draftwatermark}
%\SetWatermarkText{DRAFT}
%\SetWatermarkScale{0.5}
%\SetWatermarkColor[gray]{0.8}
%-------------------------------------------------

\begin{document} 
    
%-------------------------------------------------------------
%-- Make the header of the document                          |
%------>------>------>------>------>------>------>------>--> |
\input{includes/document-header}


%
\tableofcontents

\clearpage

\section{Potential, Field and Charge distribution}
Given 
\begin{equation} 
    V(r) = A \frac{e^{-\lambda r}}{r} \label{eq:1} 
\end{equation} 
Electric field, \(\vec{E}\):
\begin{gather}
    \vec{E} = -\vec{\nabla}V \\
    \vec{E} = -\left(\pdv{}{r}\hat{r}+\frac{1}{r \sin \phi}\pdv{}{\phi}\hat{\phi}+\frac{1}{r}\pdv{}{\phi}\hat{\phi}\right)V(r) \\
    \vec{E} = -\odv{V(r)}{r}\hat{r} 
\end{gather}
\begin{equation}
    \boxed{\vec{E} = A \frac{e^{-\lambda r}}{r^2} (1 + \lambda r)\hat{r}}
\end{equation}
Charge distribution, \(\rho\) :
\begin{gather}
    \vec{\nabla}.\vec{E} = \frac{\rho}{\epsilon_o} \\
    \vec{\nabla}.\left(A e^{-\lambda r} (1 + \lambda r)\frac{\hat{r}}{r^2}\right) = \frac{\rho}{\epsilon_o} \\
    A e^{-\lambda r} (1 + \lambda r)\vec{\nabla}.\left(\frac{\hat{r}}{r^2}\right)+ \frac{\hat{r}}{r^2}.\vec{\nabla}(A e^{-\lambda r} (1 + \lambda r))= \frac{\rho}{\epsilon _o} \\
    A e^{-\lambda r} (1 + \lambda r) (4 \pi \delta^3(r)) + \frac{\hat{r}}{r^2}.(A e^{-\lambda r}(-\lambda^2 r)\hat{r} )= \frac{\rho}{\epsilon_o} 
\end{gather}
\[
    \left(\because \vec{\nabla}.\left(\frac{\hat{r}}{r^2}\right) = 4 \pi \delta^3(r)\right)
\]
\begin{equation}
    \boxed{\rho = A\epsilon_o \left( 4 \pi \delta^3(r) -  \lambda^2\frac{e^{-\lambda r}}{r}\right)} \left(\because f(x)\delta(x)= f(0)\delta(x)\right)
\end{equation}
Total Charge, Q:
\begin{gather}
    Q = \int_{-\infty}^{+\infty} \rho d\tau\\
    Q = \int_{-\infty}^{+\infty} A\epsilon_o \left( 4 \pi \delta^3(r) -  \lambda^2\frac{e^{-\lambda r}}{r}\right) d\tau \\
    Q = A\epsilon_o \int_{-\infty}^{+\infty}4 \pi \delta^3(r) d\tau - A \epsilon_{o} \int_{-\infty}^{+\infty} \lambda^2\frac{e^{-\lambda r} }{r} d\tau \\
    Q = A \epsilon_{o} (4 \pi) - A \epsilon_{o} \lambda^2 4 \pi(\frac{1}{\lambda^2})\\
    \boxed{Q = 0}
\end{gather}
\subsection*{Answers}
\(\boxed{\vec{E} = A \frac{e^{-\lambda r}}{r^2} (1 + \lambda r)\hat{r}}\), \\ \(\boxed{\rho = A\epsilon_o \left( 4 \pi \delta^3(r) -  \lambda^2\frac{e^{-\lambda r}}{r}\right)}\) ,\\\(
    \boxed{Q = 0}\).
\newpage
\section{Dipole}
Considering Proton above Z=0 and electron below Z=0, \(10^{-11}m \ll 13m  \), so we can consider this charge distribution as dipole with dipole moment \(\vec{p} = ed \hat{k} \) 
\begin{gather}
    V(R) = \frac{q}{4 \pi \epsilon_{o} }\left(\frac{1}{R_+}-\frac{1}{R_-}\right)
\end{gather}
Law of cosines, 
\begin{gather}
    R_\pm^2 = R^2 + (\frac{d}{2})^2 \mp Rd\cos \phi = R^2(1\mp \frac{d}{R}\cos\phi + \frac{d^2}{4 R^2})\\
    \frac{1}{R_\pm} \approx \frac{1}{R}\left(1\mp \frac{d}{R}\cos\phi\right)^{-\frac{1}{2}} \approx \frac{1}{R}\left(1\pm \frac{d}{2R}\cos\phi\right)\\
    \implies \frac{1}{R_+} - \frac{1}{R_-} \approx \frac{d}{
        R^2}\cos\phi
\end{gather}

\begin{equation}
    V(R) \cong \frac{1}{4\pi \epsilon_{o} } \frac{qd\cos\phi}{R^2}
\end{equation}
\begin{equation}
    \boxed{V(R) = \frac{1}{4\pi \epsilon_{o} } \frac{\vec{p}.\hat{R}}{R^2}}
\end{equation}
Electric Field, \(\vec{E}(R)\) is
\begin{gather}
    \vec{E}(R) = -\vec{\nabla}V(R)\\
    \vec{E}(R) = -\left(\pdv{}{R}\hat{R} +\frac{1}{R \sin \phi}\pdv{}{\theta}+\frac{1}{R}\pdv{}{\phi}\right)\left(\frac{1}{4\pi \epsilon_{o} }\frac{qd\cos\phi}{R^2}\right)\\
    \vec{E}(R) = \frac{qd}{4\pi \epsilon_{o} R^3} \left(2 \cos\phi \hat{R} + \sin \phi \hat{\phi}  \right)
\end{gather}
\begin{equation}
    \boxed{\vec{E}(R) = \frac{1}{4 \pi \epsilon_{o} R^3}\left(3(\vec{p}.\hat{R})\hat{R} -\vec{p} \right)}
\end{equation}
Here, \(\vec{p} = 10^{-11} e \hat{k} \)Cm  and \(\vec{R} = (3 \hat{i} + 4 \hat{i}+12 \hat{k})m \)  
\begin{gather}
    V(R) = \frac{1}{4\pi \epsilon_{o} } \frac{12 X 10^{-11} e }{13^3}V = 4.65 X 10^{-25} V  \\
    \vec{E} (R)= \frac{1}{4 \pi \epsilon_{o} 13^3}(3(\frac{12}{13}X10^{-11}e)\frac{3 \hat{i} + 4 \hat{j} +12 \hat{k} }{13} - 10^{-11}e \hat{k}   )\\
    \vec{E}(R) = (4.188X10^{-24}\hat{i}+ 5.585X10^{-24} \hat{j}+1.019 X 10^{-23} \hat{k} )NC^{-1}  
\end{gather}
\subsection*{Answers}
\(\boxed{V(R) =4.65 X 10^{-25} V}\\\boxed{\vec{E}(R)=  (4.188X10^{-24}\hat{i}+ 5.585X10^{-24} \hat{j}+1.019 X 10^{-23} \hat{k} )NC^{-1} }\)
\section{Magnetic Boundary}
\subsection*{Boundary Conditions}
\subsubsection*{Normal Component}
\begin{figure}[H]
    \centering
    \incfig{normcom}
    \caption{Pill box}
    \label{fig:normcom}
\end{figure}
\begin{gather}
    \oint \vec{B}. d\vec{s} = 0 \\
    B_{1n} \Delta S - B_{2n} \Delta S = 0 \implies B_{1n} = B_{2n}
\end{gather}
\subsubsection*{Tangential Component}
\begin{figure}[H]
    \centering
    \incfig{magtan}
    \caption{Loop}
    \label{fig:magtan}
\end{figure}
\begin{gather}
    \oint \vec{H}.d\vec{l} = I_{f}\\
    H_{1t}\Delta W - H_{2t} \Delta W = \Delta W k_s\\
    H_{1t} - H_{2t}=k_s 
\end{gather}
Assuming source free boundary, $k_s =0$. 
\begin{gather}
    B_{1n} = B_{2n}\\
    H_{1t} = H_{2t}  \implies \frac{B_{1t}}{\mu_1} =  \frac{B_{2t}}{\mu_2} 
\end{gather}
\subsection*{(i) \(\mathbf{\vec{B_1} = 0.5 \hat{x} -10 \hat{y} (mT)}\) }
Boundary is y = 0 therefore, normal is \(\hat{y} \) 
\begin{gather}
    B_{1n} = B_{2n}\implies B_{2n} = -10mT \\
    \frac{B_{1t}}{\mu_1} = \frac{B_{2t}}{\mu_2} \implies  B_{2t} = \frac{\mu_2}{\mu_1}B_{1t}\\
    B_{2t} = 2.5 T\\
    \text{Angle with interface is} \arctan (B_{2n}/B_{2t} ) = \arctan (-\frac{10*10^{-3} }{2.5}) = 0.114^{\circ}
\end{gather}
\begin{equation}
    \boxed{\vec{B_2} = 2.5T\hat{x}  - 10mT \hat{y} , \text{makes } 0.114^{\circ} \text{ with interface.}}
\end{equation}
\subsection*{(ii) \(\mathbf{\vec{B_2} = 10 \hat{x} + 0.5 \hat{y} (mT)}\) }
Boundary is y = 0 therefore, normal is \(\hat{y} \) 
\begin{gather}
    B_{1n}  = B_{2n} \implies B_{1n} = 0.5 mT\\
    \frac{B_{1t}}{\mu_1} = \frac{B_{2t}}{\mu_2} \implies B_{1t} = \frac{\mu_1}{\mu_2}B_{2t}\\
    B_{1t} = \frac{10}{5000}mT = 2 \mu T\\
    \text{Angle with normal is} \arctan(B_{1t}/B_{1n}) = \arctan(\frac{0.002}{0.5}) = 0.229^{\circ}
\end{gather}
\begin{equation}
    \boxed{\vec{B_1} = 2\mu T\hat{x}  +0.5mT \hat{y} , \text{makes } 0.229^{\circ} \text{ with normal.}}
\end{equation}
\section{Maxwell's equations in scalar form}
\subsection*{Maxwell equations}
\begin{gather}  
   \vec{\nabla}.\vec{D} = \rho\\
   \vec{\nabla}. \vec{B} = 0 \\
   \vec{\nabla}X\vec{E}  = -\pdv{B}{t}\\
   \vec{\nabla}X\vec{H} = \vec{J}+\pdv{D}{t}
\end{gather}
\subsection*{Linear Medium}
\begin{gather}
    \vec{D} = \epsilon \vec{E}\\
    \vec{B} = \mu \vec{H}
\end{gather}
\subsection*{(i) Cartesian coordinates}
\begin{gather}
    \vec{\nabla}.\vec{D} = \pdv{\epsilon E_{x} }{x}+\pdv{\epsilon E_{y} }{y}+\pdv{\epsilon E_{z} }{z} = \rho \\
    \vec{\nabla}.\vec{B} = \pdv{B_{x} }{x}+\pdv{B_{y} }{y}+\pdv{B_{z} }{z} = 0\\
    \vec{\nabla}X\vec{E} = \left(\pdv{E_{z} }{y}-\pdv{E_{y} }{z}\right)\hat{i} +  \left(\pdv{E_{x} }{z}-\pdv{E_{z} }{x}\right)\hat{j} + \left(\pdv{E_{y} }{x}-\pdv{E_{x} }{y}\right)\hat{k} = -\pdv{B_{x} }{t}\hat{i} -\pdv{B_{y} }{t}\hat{j} -\pdv{B_{z} }{t}\hat{k} \\
    \begin{split}
        \vec{\nabla}X\vec{H} = \frac{1}{\mu}\left({\left(\pdv{B_{z} }{y}-\pdv{B_{y} }{z}\right)\hat{i} +  \left(\pdv{B_{x} }{z}-\pdv{B_{z} }{x}\right)\hat{j} + \left(\pdv{B_{y} }{x}-\pdv{B_{x} }{y}\right)\hat{k}}\right)= \\ J_x \hat{i} + J_y \hat{j} + J_{z} \hat{k} + 
        \epsilon (\pdv{E_{x} }{t}\hat{i} +\pdv{E_{y} }{t}\hat{j} +\pdv{E_{z} }{t}\hat{k})
    \end{split}  
\end{gather}
\subsubsection*{Scalar equations}
\begin{equation}
    \boxed{\pdv{ E_{x} }{x}+\pdv{ E_{y} }{y}+\pdv{ E_{z} }{z} = \frac{\rho}{\epsilon}}
\end{equation}
\begin{equation}
    \boxed{\pdv{B_{x} }{x}+\pdv{B_{y} }{y}+\pdv{B_{z} }{z} = 0}
\end{equation}
\begin{equation}
    \boxed{\left(\pdv{E_{z} }{y}-\pdv{E_{y} }{z}\right)+\pdv{B_{x} }{t}=0}
\end{equation}
\begin{equation}
    \boxed{ \left(\pdv{E_{x} }{z}-\pdv{E_{z} }{x}\right)+\pdv{B_{y} }{t}=0}
\end{equation}
\begin{equation}
    \boxed{\left(\pdv{E_{y} }{x}-\pdv{E_{x} }{y}\right)+\pdv{B_{z} }{t}=0}
\end{equation}
\begin{equation}
    \boxed{\left(\pdv{B_{z} }{y}-\pdv{B_{y} }{z}\right)=\mu J_x + \mu\epsilon \pdv{E_x}{t}}
\end{equation}
\begin{equation}
    \boxed{\left(\pdv{B_{x} }{z}-\pdv{B_{z} }{x}\right)= \mu J_y + \mu\epsilon \pdv{E_y}{t}}
\end{equation}
\begin{equation}
    \boxed{\left(\pdv{B_{y} }{x}-\pdv{B_{x} }{y}\right)=\mu J_z + \mu\epsilon \pdv{E_z}{t} }
\end{equation}
\subsection*{(ii) Cylindrical Coordinates}
\begin{gather}
    \frac{1}{r}\pdv{(rE_r)}{r}+\frac{1}{r}\pdv{E_\phi}{\phi}+\pdv{E_{z} }{z} = \frac{\rho}{\epsilon }\\
    \frac{1}{r}\pdv{(rB_r)}{r}+\frac{1}{r}\pdv{B_\phi}{\phi}+\pdv{B_z}{z} = 0\\
    \begin{split}
    \left(\frac{1}{r}\pdv{E_z}{\phi}-\pdv{E_\phi}{z} \right)\hat{r} + \left(\pdv{E_r}{z}-\pdv{E_z}{r} \right)\hat{\phi} +\frac{1}{r}\left(\pdv{(rE_\phi)}{r}-\pdv{E_r}{\phi}\right)\hat{z} =\\ -\pdv{B_r}{t}\hat{r} -\pdv{B_\phi}{t}\hat{\phi} -\pdv{B_z}{t}\hat{z} 
    \end{split}\\
    \begin{split}
        \left(\frac{1}{r}\pdv{B_z}{\phi}-\pdv{B_\phi}{z} \right)\hat{r} + \left(\pdv{B_r}{z}-\pdv{B_z}{r} \right)\hat{\phi} +\frac{1}{r}\left(\pdv{(rB_\phi)}{r}-\pdv{B_r}{\phi}\right)\hat{z} = \\ \mu (J_{r} \hat{r} + J_\phi \hat{\phi} +J_z \hat{z} )+ \mu \epsilon (\pdv{E_{r} }{t}\hat{r} +\pdv{E_{\phi} }{t}\hat{\phi} +\pdv{E_{z} }{t}\hat{z})
    \end{split}
\end{gather}
\subsubsection*{Scalar Equations}
\begin{equation}
\boxed{\frac{1}{r}\pdv{(rE_r)}{r}+\frac{1}{r}\pdv{E_\phi}{\phi}+\pdv{E_{z} }{z} = \frac{\rho}{\epsilon }}
\end{equation}
\begin{equation}
\boxed{\frac{1}{r}\pdv{(rB_r)}{r}+\frac{1}{r}\pdv{B_\phi}{\phi}+\pdv{B_z}{z} = 0}
\end{equation}
\begin{equation}
\boxed{\frac{1}{r}\pdv{E_z}{\phi}-\pdv{E_\phi}{z}+\pdv{B_r}{t} =0}
\end{equation}
\begin{equation}
\boxed{ \pdv{E_r}{z}-\pdv{E_z}{r}+\pdv{B_\phi}{t}=0 }
\end{equation}
\begin{equation}
\boxed{\pdv{(rE_\phi)}{r}-\pdv{E_r}{\phi}+\pdv{B_z}{t}=0}
\end{equation}
\begin{equation}
\boxed{\frac{1}{r}\pdv{B_z}{\phi}-\pdv{B_\phi}{z}=\mu J_r + \mu \epsilon \pdv{E_r}{t}}
\end{equation}
\begin{equation}
\boxed{\pdv{B_r}{z}-\pdv{B_z}{r}=\mu J_\phi + \mu \epsilon \pdv{E_\phi}{t}}
\end{equation}
\begin{equation}
\boxed{\pdv{(rB_\phi)}{r}-\pdv{B_r}{\phi}=\mu J_z + \mu \epsilon \pdv{E_z}{t}}
\end{equation}
\subsection*{(iii) Spherical coordinates}
\begin{gather}
    \frac{1}{r^2}\pdv{(r^2 E_r)}{r}+\frac{1}{r \sin \theta}\pdv{E_\theta \sin \theta}{\theta}+\frac{1}{r \sin \theta}\pdv{E_\phi}{\phi}=\frac{\rho}{\epsilon }\\
    \frac{1}{r^2}\pdv{(r^2 B_r)}{r}+\frac{1}{r \sin \theta}\pdv{B_\theta \sin \theta}{\theta}+\frac{1}{r \sin \theta}\pdv{B_\phi}{\phi}=0\\
    \begin{split}
    \frac{1}{r \sin \theta}\left(\pdv{E_\phi \sin \theta}{\theta}-\pdv{E_\theta}{\phi}\right)\hat{r} + \frac{1}{r}\left(\frac{1}{\sin \theta}\pdv{E_r}{\phi}-\pdv{(rE_\phi)}{r}\right)\hat{\theta} +\frac{1}{r}\left(\pdv{(rE_\theta)}{r}-\pdv{E_r}{\theta}\right)\hat{\phi}=\\-\pdv{B_r}{t}\hat{r}-\pdv{B_\theta}{t}\hat{\theta}  -\pdv{B_\phi}{t}\hat{\phi} 
    \end{split}\\
    \begin{split}
        \frac{1}{r \sin \theta}\left(\pdv{B_\phi \sin \theta}{\theta}-\pdv{B_\theta}{\phi}\right)\hat{r} + \frac{1}{r}\left(\frac{1}{\sin \theta}\pdv{B_r}{\phi}-\pdv{(rB_\phi)}{r}\right)\hat{\theta} +\frac{1}{r}\left(\pdv{(rB_\theta)}{r}-\pdv{B_r}{\theta}\right)\hat{\phi}=\\\mu (J_{r} \hat{r}+J_\theta \hat{\theta} + J_\phi \hat{\phi})+\mu \epsilon \left(\pdv{E_r}{t}\hat{r}+\pdv{E_\theta}{t}\hat{\theta}  +\pdv{E_\phi}{t}\hat{\phi} \right)
        \end{split} 
\end{gather}
\subsection*{Scalar equations}
\begin{equation}
    \boxed{\frac{1}{r^2}\pdv{(r^2 E_r)}{r}+\frac{1}{r \sin \theta}\pdv{E_\theta \sin \theta}{\theta}+\frac{1}{r \sin \theta}\pdv{E_\phi}{\phi}=\frac{\rho}{\epsilon }}
\end{equation}
\begin{equation}
    \boxed{\frac{1}{r^2}\pdv{(r^2 B_r)}{r}+\frac{1}{r \sin \theta}\pdv{B_\theta \sin \theta}{\theta}+\frac{1}{r \sin \theta}\pdv{B_\phi}{\phi}=0}
\end{equation}
\begin{equation}
    \boxed{ \frac{1}{r \sin \theta}\left(\pdv{E_\phi \sin \theta}{\theta}-\pdv{E_\theta}{\phi}\right)+\pdv{B_r}{t}=0}
\end{equation}
\begin{equation}
    \boxed{\frac{1}{r}\left(\frac{1}{\sin \theta}\pdv{E_r}{\phi}-\pdv{(rE_\phi)}{r}\right)+\pdv{B_\theta}{t}=0}
\end{equation}
\begin{equation}
    \boxed{\frac{1}{r}\left(\pdv{(rE_\theta)}{r}-\pdv{E_r}{\theta}\right)+\pdv{B_\phi}{t}=0}
\end{equation}
\begin{equation}
    \boxed{\frac{1}{r \sin \theta}\left(\pdv{B_\phi \sin \theta}{\theta}-\pdv{B_\theta}{\phi}\right)= \mu J_r + \mu \epsilon \pdv{E_r}{t}}
\end{equation}
\begin{equation}
    \boxed{\frac{1}{r}\left(\frac{1}{\sin \theta}\pdv{B_r}{\phi}-\pdv{(rB_\phi)}{r}\right)=\mu J_\theta + \mu \epsilon \pdv{E_\theta}{t}}
\end{equation}
\begin{equation}
    \boxed{\frac{1}{r}\left(\pdv{(rB_\theta)}{r}-\pdv{B_r}{\theta}\right) = \mu J_\phi + \mu \epsilon \pdv{E_\phi}{t}}
\end{equation}
\section{Lorentz Condition and Equation of Continuity}
A \textbf{corollary}  of \textbf{Helmholtz} \textbf{Decomposition} \textbf{theorem} says that all physically realistic scalar fields obey a continuity equation. The theorem states that for any reasonable scalar field S and Vector field \textbf{C}  there exists a vector field \textbf{F} such that \(\vec{\nabla}\).\textbf{F} = S and\(\vec{\nabla}\)X\textbf{F} = \textbf{C}. \href{http://dfcd.net/articles/potentialfields.pdf}{reference}\newline
\textbf{Lorentz Gauge}: 
\begin{gather}
    \vec{\nabla}.\vec{A} = -\frac{1}{c^2}\pdv{\phi}{t}
\end{gather}
from definition  of Magnetic Vector potential, \(\vec{A}\)
\begin{equation}
    \vec{\nabla}X\vec{A} = \vec{B}
\end{equation}
Considering \textbf{F} = $\vec{A}$, S = $-\frac{1}{c^2}\pdv{\phi}{t}$, \textbf{C} = \(\vec{B}\) \\
Lorentz condition satisfy the condition for equation of continuity. 
\section{Homogenous wave equation}
\begin{gather}
    U = f(t \pm R\sqrt{\mu \epsilon } )
\end{gather}
Let \(x = t \pm R\sqrt{\mu \epsilon }\)
\begin{gather}
    \pdv{U}{R} = \odv{f}{x} \pdv{x}{R}\\
    \pdv{U}{R} = \pm\sqrt{\mu \epsilon} \odv{f}{x}
\end{gather}
\begin{equation}
    \boxed{\frac{\partial ^2 U}{\partial R^2} = \mu \epsilon \frac{d^{2}f }{dx^2}}
\end{equation}
\begin{gather}
    \pdv{U}{t} = \odv{f}{x} \pdv{x}{t}\\
    \pdv{U}{t} = \odv{f}{x}
\end{gather}
\begin{equation}
    \boxed{\frac{\partial ^2 U}{\partial t^2} = \frac{d^{2}f }{dx^2}}
\end{equation}
\begin{equation}
    \boxed{\frac{\partial ^2 U}{\partial R^2} -\frac{\partial ^2 U}{\partial t^2} = \mu \epsilon \frac{d^{2}f }{dx^2} - \mu \epsilon \frac{d^2 f}{dx^2} = 0}
\end{equation}
Therefore, any function of $t\pm R \sqrt{\mu \epsilon}$ satisfies the Homogenous wave equation.  
\end{document} 
