\chapter{Electrodynamics(Review)}
\lecture{1}{25 July 2022}{First Lecture}
\section{Electrostatics(Review)}
Two postulates(propositions) in Electrostatics
\begin{proposition}
	\[
		\vec{\nabla} . \vec{E} = \frac{\rho}{\epsilon_{o} }
	\]
	\[
		\oint_s {\vec{E}.\vec{ds}} = \frac{Q_{en} }{\epsilon _o}
	\]
\end{proposition}
\begin{proposition}
	\[
		\vec{\nabla} X \vec{E} = \vec{0}
	\]
	\[
		\oint_c {\vec{E}.\vec{dl}} = 0
	\]
\end{proposition}
Electric field around a closed path is 0.
%────────────────────────────────────────────────────────────────────────────────────────────────────────────────────────────────────────────────────
\begin{theorem}
	kirchhoff's voltage Law, Algebraic sum of voltage drop in a closed loop is zero.
\end{theorem}

\begin{theorem}[Coloumbs Law]
\(\vec{E}\) due to a point charge "q" is 
\[
	\vec{E} = \frac{1}{4 \pi \epsilon_o} \frac{q}{\left\vert \vec{R} \right\vert ^3}\vec{R} 
\]
For a discrete distribution of charges
\[
	\vec{E} = \frac{1}{4 \pi \epsilon_o} \sum_{j=1}^{n} \frac{q_j}{\left\vert \vec{R}-\vec{R_j} \right\vert ^3}(\vec{R}-\vec{R_j})
\]
For a continuos distribution of charges
\[
	\vec{E} = \frac{1}{4 \pi \epsilon_o} \int \hat{R} \frac{\rho}{\vec{R}^2} dV^\prime 
\]
\end{theorem}

since, 
\[
	\vec{\nabla} X \vec{E} = \vec{0}
\]

\[
	\vec{E} = - \vec{\nabla}V
\]
Negative sign because potential decreases in the direction of electric field by convention.

\begin{definition}
	\[
	\vec{E} = - \vec{\nabla}V
\]	
\[
	V_2 - V_1 = \int _{P_1}^{P_2} \vec{E}.\vec{dl}
\]
\[
	V(R) = \frac{1}{4 \pi \epsilon_o} \int \frac{\rho}{R} dV^\prime  
\]
prime indicates coordinate system w.r.t source.
\end{definition}

\begin{definition}[Dipole]
	A pair of equal and opposite charges separated by a distance.
\end{definition}
\[
		V_{p} (R) = \frac{q}{4 \pi \epsilon_o}[\frac{1}{R_+}-\frac{1}{R_-}]
\]
\[		
		V_{p} (R) = \frac{q d \cos \theta}{4 \pi \epsilon_o R^2} = \frac{\vec{p}.\hat{R}}{4 \pi \epsilon_o R^2}
\]
\[
	\vec{E_{p}} (R) = -\vec{\nabla}V_{p} (R)
\]
\[
	\vec{E_{p}} (R) = [-\hat{R} \frac{\partial}{\partial R} - \frac{\hat{\theta}}{R} \frac{\partial}{\partial \theta}]V_{p}(R)
\]
\[
	\vec{E} = \frac{\vec{p}}{4 \pi \epsilon_{o} R^3 }[2 \cos \theta \hat{R} + \sin \theta \hat{\theta}]
\]
\[
	\vec{E} = \frac{1}{4 \pi \epsilon_{o} R^3 } [3 (\vec{p}.\hat{R})\hat{R} - \vec{p}]
\]
\subsection*{Conductor}
\(\vec{E} = 0 \) inside a conductor.

%─────────Gauss law picture ───────────────────────────────────────────────────────────────────────────────────────────────────────────────────────────────────────────

\[
	\oint_c \vec{E}.\vec{dl} = 0
\]
\[
	\vec{E_{t1}} . \Delta \vec{w} - \vec{E_{t2}} \Delta \vec{w} = 0
\]
\(\vec{E_{t1}} = 0  \because\) inside a conductor.\\ \\
\(\therefore \vec{E_{t2}} = 0\).\\
\(\therefore\)  No tangential electric field.

%─────Normal electric field in conductor diagram ───────────────────────────────────────────────────────────────────────────────────────────────────────────────────────────────────────────────
\[
	\frac{\sigma_s \Delta s}{\epsilon_{o}} = [\vec{E_1}.\vec{n_1}+\vec{E_{2} }.\vec{n_2}] \Delta s
\]
\[
	\vec{E_2} = 0
\] inside a conductor.
\[
	\vec{E_1} = \frac{\sigma_s}{\epsilon_{o}} \hat{n}
\]
Normal electric field is \(\vec{E_n} = \frac{\sigma_s}{\epsilon_{o}} \hat{n}\) 

\subsection*{Dielectric in a static $\vec{E}$}
\[
	\vec{p} = q \vec{d}
\]
\[
	\vec{\tau} = \vec{p} X \vec{E}
\]
\[
	U = - \vec{p} . \vec{E}
\]
Gauss law in differential form
\[
	\vec{\nabla}.\vec{E} = \frac{\rho}{\epsilon_{o}} = \frac{1}{\epsilon_o} [\rho_{f} +\rho_{b}]
\]
\begin{definition}
	The \emph{polarization} of a medium P gives the electric dipole moment per unit volume of the material.
	\[
		\vec{P} = \frac{n \vec{p}}{V}
	\]
\end{definition}
Bound charges
\[
	\sigma_b = -\vec{P}.\hat{n}
\]
\[
	\rho_{b} = -\vec{\nabla}.\vec{P}
\]
Potential V,
\[
	V = \frac{1}{4 \pi \epsilon_{o}} \oint \frac{\vec{P}.\hat{n}}{R} ds + \frac{1}{4 \pi \epsilon_{o}} \int \frac{-\vec{\nabla}.\vec{P}}{R}dV
\]
refer David J griffith's for detailed explanation.

For a Dielectric,
\[
	\vec{\nabla}.\vec{E} = \frac{1}{\epsilon_{o}}(\rho_f - \vec{\nabla}.\vec{P})
\]
\[
	\vec{\nabla}.[\epsilon_{o} \vec{E}+\vec{P}] = \rho_f
\]
\begin{definition}
	Displacement vector, \(\vec{D} = \epsilon_{o} \vec{E}+\vec{P}\) 
	\[
		\vec{\nabla}.\vec{D} = \rho_f
	\]
\end{definition}