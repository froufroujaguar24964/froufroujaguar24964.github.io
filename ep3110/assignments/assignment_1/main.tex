\documentclass[11pt, a4paper]{article}

% Set the title of the current document to be produced.
\newcommand{\doctitle}{Assignment I}
% Command for the due date of the homework.
\newcommand{\duedate}{\color{rltblue}{\faCalendarCheckO { }Due date: August 11th, before midnight \faCalendarCheckO	}}
\newcommand{\myname}{\color{rltred} {}\textbf{Chaganti}  \textbf{Kamaraja}  \textbf{Siddhartha} }
\newcommand{\rollnumber}{\color{rltred}{}\textbf{EP20B012} }
%------------------------------------------------------------
% Import commands for both teacher and course information.  | 
% NOTE: Change your teacher and course info in these files. |
%------>------>------>------>------>------>------>------>-->|
%-------------------------------------------------
% Teacher-specific commands                      |
%---------------                                 |
%-> Instructions: change your teacher info here. |
%------->------>------>------>------>------>---->|
%
\newcommand{\instructor}{James Libby}
\newcommand{\office}{HSB 116A}
\newcommand{\hours}{By appointment}
\newcommand{\phone}{044 - 2257 4885}
\newcommand{\college}{IIT Madras}
\newcommand{\email}{Libby@iitm.ac.in}
\newcommand{\faculty}{Professor}
\newcommand{\department}{Physics}
                              %|
%-------------------------------------------------
% Course-specific commands                       |
%---------------                                 |
%-> Instructions: change your course info here.  |
%------->------>------>------>------>------>---->|
%
\newcommand{\semester}{July-Nov 2022}
\newcommand{\csection}{00001 \& 00002}
\newcommand{\ponderation}{2-4-3 (Theory-Lab-Homework)}
\newcommand{\coursetitle}{High Energy Physics}
\newcommand{\coursenumber}{PH5211}
\newcommand{\prerequisite}{All porgram courses semesters 1-4}
                               %|   
%
%------------------------------------------------------------
%-- Import packages and custom command definitons.          |
%------>------>------>------>------>------>------>------>-->|
%----------------------------------------------------
% The following is a list of LaTeX packages imports |
%------->------>------>------>------>------>---->---|
%

% The margins at the bottom of the page has been reduced.
% this allows for a slim footer.
\usepackage[left=1in,right=1in,top=1in,bottom=0.7in]{geometry}
% Original size:
%\usepackage[inner=1.5cm,outer=1.5cm,top=1.5cm,bottom=.5cm,margin=1in]{geometry}
\usepackage[
    colorlinks,
    pagebackref,
    pdfusetitle,
    urlcolor=blue,
    citecolor=blue,
    linkcolor=blue,    
    plainpages=false]
{hyperref}            
% ftp://ftp.dante.de/tex-archive/fonts/bbding/bbding.pdf
%https://ctan.math.illinois.edu/fonts/bbding/bbding.pdf
\usepackage{fancyhdr, lastpage, bbding, pmboxdraw}
\usepackage{fancyvrb}
\PassOptionsToPackage{usenames,dvipsnames}{xcolor}
\usepackage{acronym}
\usepackage{amsthm}
\usepackage{caption}
\usepackage{xcolor}
\usepackage{enumitem}
\usepackage{tabularx}
\usepackage{sectsty}
\usepackage{amssymb}
% pifont package doc at: https://ctan.math.ca/tex-archive/macros/latex/required/psnfss/psnfss2e.pdf
% pifont is used to define custom list and style list items using the \ding command. 
\usepackage{pifont} 
% bclogo used for making a colored box for notes. 
% @see: https://ctan.org/pkg/bclogo?lang=en
\usepackage[tikz]{bclogo} 
\usepackage{titlesec}  
\usepackage[open,openlevel=1]{bookmark}

%-- @see http://ctan.sharelatex.com/tex-archive/fonts/fontawesome/doc/fontawesome.pdf
% Font Awesome  http://ctan.math.washington.edu/tex-archive/fonts/fontawesome5/doc/fontawesome5.pdf
% https://muug.ca/mirror/ctan/fonts/fontawesome5/doc/fontawesome5.pdf
\usepackage{fontawesome5}
\usepackage{fontawesome}
%---------------------------------
% ==== Font setup.
% Load any of the following fonts.
%---------------------------------
%\usepackage{lmodern}
%\usepackage{mathptmx}
%\usepackage{times}
%\usepackage[sc]{mathpazo} % Palatino font.
%\linespread{1.05} % Palatino needs more leading (space between lines)
\usepackage{tgbonum} % For Bonum/Bookman font.
\usepackage[utf8]{inputenc}
\usepackage[T1]{fontenc}
%---------------------------------
\usepackage{booktabs} 

\pagestyle{empty}
\usepackage{graphicx}
\usepackage{multicol}
\usepackage{blindtext}  
\usepackage{vhistory} % for making a table for the revision history.
                                  %|  
%--------------------------------------------------------
%--> \customhrule: makes a customized rule whose width  | 
%                  should be passed as parameter.       |
%--------------------------------------------------------
\newcommand{\customhrule}[1]{
	\rule[1.4pt]{\linewidth}{#1}
}
%------------------------------------------------------
%--> \doublerule: makes a double rule.                |
%------------------------------------------------------ 
\newcommand{\doublerule}[1][.4pt]{
	\noindent
	\makebox[0pt][l]{\rule[.7ex]{\linewidth}{#1}}%
	\rule[1pt]{\linewidth}{#1}\par} 
%===== Custom Ruler commands  ==================
\renewcommand{\headrulewidth}{1pt}
\renewcommand{\footrulewidth}{0.4pt}

% Disable spaces between list items in a labeled list.
\setlist{noitemsep}
 
%-------------------------------------------------------------
%= The followig are declaraions of custom Lists              =
%-------------------------------------------------------------
%
%======= Green rectangles list =======================
% \Rectangle from bbind
\newlist{greenrectangles}{itemize}{4}
%\setlist[greenrectangles]{topsep=4pt,partopsep=0pt,itemsep=3pt,parsep=0pt,labelindent=0.5cm,leftmargin=*}
\setlist[greenrectangles]{itemsep=5pt,parsep=0pt,topsep=4pt,partopsep=3pt}
\setlist[greenrectangles,1]{font=\color{darkred},label={\color{darkgreen}{\Rectangle}}}

%======= Alphabetical  list =======================
\newlist{alphalist}{enumerate}{9}
\setlist[alphalist]{topsep=4pt,partopsep=0pt,itemsep=3pt,parsep=0pt,labelindent=0.5cm,leftmargin=*}
\setlist[alphalist,1]{label=\textbf{\alph*)}}
%======= Non-numbered list =======================
\newlist{itemizedlist}{itemize}{9}
\setlist[itemizedlist]{topsep=4pt,partopsep=0pt,itemsep=3pt,parsep=0pt,labelindent=0.5cm,leftmargin=*}
%\setlist[itemizedlist,1 ]{label=\textbf{\alph*)}}

%======= Arrowed list =======================
\newlist{arrows}{itemize}{4}
\setlist[arrows]{topsep=4pt,partopsep=0pt,itemsep=3pt,parsep=0pt,labelindent=0.5cm,leftmargin=*}
\setlist[arrows,1]{font=\color{darkred},label={\HandRight}}

%======= Bordered square list =======================
% Colorize the selected symbol? 
% ❏
\newlist{borderedsquare}{itemize}{4}
\setlist[borderedsquare]{topsep=4pt,partopsep=0pt,itemsep=3pt,parsep=0pt,labelindent=0.5cm,leftmargin=*}
\setlist[borderedsquare,1]{label=\ding{111}}

%======= Filled, curved arrow list =======================
\newlist{curveddarrow}{itemize}{4}
\setlist[curveddarrow]{topsep=4pt,partopsep=0pt,itemsep=3pt,parsep=0pt,labelindent=0.5cm,leftmargin=*}
\setlist[curveddarrow,1]{label=\small\faMarker}

%======= Colored pen list ======================= 
\newlist{coloredPen}{itemize}{4}
\setlist[coloredPen]{topsep=4pt,partopsep=0pt,itemsep=3pt,parsep=0pt,labelindent=0.5cm,leftmargin=*}
\setlist[coloredPen,1]{font=\color{darkred},label=\small\faMarker}

%======= Objectives list ======================= 
% ➠
\newlist{objectives}{itemize}{4}
\setlist[objectives]{topsep=4pt,partopsep=0pt,itemsep=3pt,parsep=0pt,labelindent=0.5cm,leftmargin=*}
\setlist[objectives,1]{label=\small\ding{224}}

%======= Dark starred list ======================= 
% ✸
\newlist{filledstarlist}{itemize}{4}
\setlist[filledstarlist]{topsep=4pt,partopsep=0pt,itemsep=3pt,parsep=0pt,labelindent=0.5cm,leftmargin=*}
\setlist[filledstarlist,1]{label=\small\ding{88}}

%======= Dark-bordered empty circle list ======================= 
% ❍
\newlist{emptyCircleList}{itemize}{4}
\setlist[emptyCircleList]{topsep=4pt,partopsep=0pt,itemsep=3pt,parsep=0pt,labelindent=0.5cm,leftmargin=*}
\setlist[emptyCircleList,1]{label=\small\ding{109}}

%======= Filled right arrow list ======================= 
% ➤
\newlist{filledRightArrowList}{itemize}{4}
\setlist[filledRightArrowList]{topsep=4pt,partopsep=0pt,itemsep=3pt,parsep=0pt,labelindent=0.5cm,leftmargin=*}
\setlist[filledRightArrowList,1]{label=\small\ding{228}}

%======= Numbered list: non-filled circle list ======================= 
% ➀
\newlist{numberedEmptyList}{itemize}{9}
\setlist[numberedEmptyList]{topsep=4pt,partopsep=0pt,itemsep=3pt,parsep=0pt,labelindent=0.5cm,leftmargin=*}
\setlist[numberedEmptyList,9]{label=\ding{182}}

%======= Right hand pointing list =======================
\newlist{rightHandPointingList}{itemize}{4}
\setlist[rightHandPointingList]{topsep=4pt,partopsep=0pt,itemsep=3pt,parsep=0pt,labelindent=0.5cm,leftmargin=*}
\setlist[rightHandPointingList,1]{font=\color{darkred},label={\HandRight}}

%----------------------------------------------------------------------
%=   The followig are custom colors declaraions                       |
%--  more colors codes can be found at: http://latexcolor.com/        | 
%-- usage: {\color{declared-color} some text}.                        |    
%  e.g.,: {\color{darkblue}{ This text will appear darkblue-colored}} |
%----------------------------------------------------------------------
\definecolor{darkblue}{rgb}{0,0,.6}
\definecolor{darkred}{rgb}{.7,0,0}
\definecolor{darkgreen}{rgb}{0,.6,0}
\definecolor{darkestred}{rgb}{.8,.1,0}
\definecolor{red}{rgb}{.98,0,0}
\definecolor{OliveGreen}{cmyk}{0.64,0,0.95,0.40}
\definecolor{CadetBlue}{cmyk}{0.62,0.57,0.23,0}
\definecolor{lightlightgray}{gray}{0.93}
\definecolor{vanierred}{RGB}{210,0,2}
\definecolor{darkestblue}{rgb}{0.0, 0.0, 0.55}
\definecolor{darkblue}{rgb}{0,0,.6}
\definecolor{darkred}{rgb}{.7,0,0}
\definecolor{darkgreen}{rgb}{0,.6,0}
\definecolor{darkestred}{rgb}{.8,.1,0}
\definecolor{red}{rgb}{.98,0,0}
\definecolor{OliveGreen}{cmyk}{0.64,0,0.95,0.40}
\definecolor{CadetBlue}{cmyk}{0.62,0.57,0.23,0}
\definecolor{lightlightgray}{gray}{0.93}
\definecolor{darkorange}{rgb}{255,140,0}
\definecolor{fluorescentyellow}{rgb}{0.8, 1.0, 0.0}
\definecolor{darkyellow}{rgb}{1,1,0.34}
\definecolor{lightyellow}{rgb}{1,1,0.6}
\definecolor{coolblack}{rgb}{0.0, 0.18, 0.39}
\definecolor{lightgray}{rgb}{.9,.9,.9}
\definecolor{darkgray}{rgb}{.4,.4,.4}
\definecolor{purple}{rgb}{0.65, 0.12, 0.82}
\definecolor{gray}{rgb}{0.4,0.4,0.4}
\definecolor{cyan}{rgb}{0.0,0.6,0.6}
\definecolor{dkgreen}{rgb}{0,0.6,0}
\definecolor{gray}{rgb}{0.5,0.5,0.5}
\definecolor{mauve}{rgb}{0.58,0,0.82}
\definecolor{lightblue}{rgb}{0.0,0.0,0.9}
\colorlet{punct}{red!60!black}
\definecolor{background}{HTML}{EEEEEE}
\definecolor{delim}{RGB}{20,105,176}
\colorlet{numb}{magenta!60!black}
\definecolor{coolblack}{rgb}{0.0, 0.18, 0.39}
\definecolor{forestgreen}{rgb}{0.0, 0.27, 0.13}
\definecolor{firebrick}{rgb}{0.7, 0.13, 0.13}
\definecolor{rltred}{rgb}{0.75,0,0}
\definecolor{rltgreen}{rgb}{0,0.5,0}
\definecolor{rltblue}{rgb}{0,0,0.75}
\definecolor{indigo}{rgb}{0.0, 0.25, 0.42}
\definecolor{jazzberryjam}{rgb}{0.65, 0.04, 0.37}
\definecolor{lava}{rgb}{0.81, 0.06, 0.13}
\definecolor{royalblue}{rgb}{0.0, 0.14, 0.4}
\definecolor{prussianblue}{rgb}{0.0, 0.19, 0.33}
\definecolor{prune}{rgb}{0.44, 0.11, 0.11}
\definecolor{cerisepink}{rgb}{0.93, 0.23, 0.51}
\definecolor{oxfordblue}{rgb}{0.0, 0.13, 0.28}
\definecolor{crimsonglory}{rgb}{0.75, 0.0, 0.2}
\definecolor{fireenginered}{rgb}{0.81, 0.09, 0.13}

%============================
% Commands for inserting colored text.
\newcommand{\bluetext}[1]{\textcolor{darkblue}{#1}}
\newcommand{\redtext}[1]{\textcolor{jazzberryjam}{#1}}

%=================================================================================================
% Command for styling tabled row header (left, center or right)
% Usage example: \thead{<Header text 1>} & \thead{<Header 2>} & \thead{<Header 3>} & \thead{<Header 4>} 
\newcommand*{\thead}[1]{\multicolumn{1}{l}{\bfseries #1}}	

%--------------------------------------------------
% ==== Doc header and footer setup.               |
%-------------------------------------------------- 
\renewcommand{\thefootnote}{\fnsymbol{footnote}}
\pagestyle{fancyplain}
\fancyhf{}
%- Disable the horizontal ruler in the header section.
\renewcommand{\headrulewidth}{0pt}
\rfoot{\fancyplain{}{page \thepage\ of \pageref{LastPage}}}
\cfoot{{\tiny{\college { } - { } \semester} }}
\lfoot{{\tiny{ \coursenumber -\coursetitle} }}
%- TODO: move the header content here.
\fancyfoot[RO, LE] {{\tiny{page \thepage\ of \pageref{LastPage} }}}
\thispagestyle{plain}
%------------------------------------------------------------

\newcolumntype{L}[1]{>{\raggedright\arraybackslash}p{#1}}
\newcolumntype{C}[1]{>{\centering\arraybackslash}p{#1}}
\newcolumntype{R}[1]{>{\raggedleft\arraybackslash}p{#1}}

%-- Spacing commands ------ 
\newcommand{\vspbpara}{\vspace*{.09in}}    
\newcommand{\customvspace}{\vspace{.5cm}}    
\titlespacing{\section}{0pt}{12pt}{9pt}
%-----
\newcommand{\vtitlespacing}{\vskip 0.3cm}
\newcommand{\paragraphentry}[1]{\noindent \textbf{\Large \underline{#1}} }
\newcommand \VRule[1][\arrayrulewidth]{\vrule width #1}
\newcommand{\bkt}[2]{\left \langle #1 \middle| #2 \right \rangle}
\newcommand{\braketmatrix}[3]{\left \langle #1 \middle| #2 \middle| #3 \right \rangle}   
%
%---> Genereate & inject metadata describing                |
%     the produced document                                 |
%--------------------------------------------------------------
%-- Set up the hyperref package.                              |
%-- Generate and inject metadate in the produced PDF document |
%------>------>------>------>------>------>------>------>-->---
 \hypersetup{pdfauthor={\instructor},%
    pdftitle={\coursenumber -- \coursetitle},%
    pdfsubject={\doctitle, Section \csection {} (\semester)},%
    pdfkeywords={\college,  \department},%
    pdfproducer={LaTeX},%
    pdfcreator={pdfLaTeX},
    bookmarks,
    bookmarksnumbered = true,
    bookmarksopen     = true,
    pdfpagelabels     = true,
    pdfstartview={XYZ null null 1.2}
}                                  %|
%------------------------------------------------------------

\topmargin      -60pt

%-----------------------------------------------------------
% Uncomment the following if you want to insert a watermark! 
%
%--> Watermark package settings: 
%\usepackage{draftwatermark}
%\SetWatermarkText{DRAFT}
%\SetWatermarkScale{0.5}
%\SetWatermarkColor[gray]{0.8}
%-------------------------------------------------

\begin{document} 
    
%-------------------------------------------------------------
%-- Make the header of the document                          |
%------>------>------>------>------>------>------>------>--> |
%--------------------------------------------------------------------------
%- The following produces the document header including the title.        |
%- The document header includes: the college/university name, faculty,    |
%  department, course number and title as well as the assignment/homework | 
%  title and due date.                                                    | 
%-------------------------------------------------------------------------|
%
\noindent % <-- need to have this first.
%
\begin{minipage}{.40\textwidth}
    {\color{darkred} \faSchool} { \textsc{\college}}{ } {\color{darkred} \faSchool}\\ 
    \small\textsc{Physics}
\end{minipage}%
\hfill	
\begin{minipage}{0.60\textwidth}%
    \raggedleft%
    {\Large \textsc{\coursenumber { } \coursetitle}\par}
    \doublerule % insert a double rule.
    \textsc{Teacher}: \instructor\\
\end{minipage}%
\vspace{2.8cm}
{
    \vspace{.3cm}
    \centering \large\myname \\
}
{
    \vspace{.2cm}
    \centering \large\rollnumber\\
}
{
    \vspace{.3cm}
    %--> Insert homework title and due date.
    \hrule\vspace{.2cm}
    \centering
    {\scshape 
        \Large \color{darkestblue}{\doctitle}{ }\textemdash{ }\small\bfseries\textsc{\semester}\par}
    \vspace{.3cm}    
}
{
    \hrule\vspace{.3cm}
    \centering  \small\duedate \\ 

}    

\vspace{3.5cm}



%
\tableofcontents

\clearpage

\section{Potential, Field and Charge distribution}
Given 
\begin{equation} 
    V(r) = A \frac{e^{-\lambda r}}{r} \label{eq:1} 
\end{equation} 
Electric field, \(\vec{E}\):
\begin{gather}
    \vec{E} = -\vec{\nabla}V \\
    \vec{E} = -\left(\pdv{}{r}\hat{r}+\frac{1}{r \sin \phi}\pdv{}{\phi}\hat{\phi}+\frac{1}{r}\pdv{}{\phi}\hat{\phi}\right)V(r) \\
    \vec{E} = -\odv{V(r)}{r}\hat{r} 
\end{gather}
\begin{equation}
    \boxed{\vec{E} = A \frac{e^{-\lambda r}}{r^2} (1 + \lambda r)\hat{r}}
\end{equation}
Charge distribution, \(\rho\) :
\begin{gather}
    \vec{\nabla}.\vec{E} = \frac{\rho}{\epsilon_o} \\
    \vec{\nabla}.\left(A e^{-\lambda r} (1 + \lambda r)\frac{\hat{r}}{r^2}\right) = \frac{\rho}{\epsilon_o} \\
    A e^{-\lambda r} (1 + \lambda r)\vec{\nabla}.\left(\frac{\hat{r}}{r^2}\right)+ \frac{\hat{r}}{r^2}.\vec{\nabla}(A e^{-\lambda r} (1 + \lambda r))= \frac{\rho}{\epsilon _o} \\
    A e^{-\lambda r} (1 + \lambda r) (4 \pi \delta^3(r)) + \frac{\hat{r}}{r^2}.(A e^{-\lambda r}(-\lambda^2 r)\hat{r} )= \frac{\rho}{\epsilon_o} 
\end{gather}
\[
    \left(\because \vec{\nabla}.\left(\frac{\hat{r}}{r^2}\right) = 4 \pi \delta^3(r)\right)
\]
\begin{equation}
    \boxed{\rho = A\epsilon_o \left( 4 \pi \delta^3(r) -  \lambda^2\frac{e^{-\lambda r}}{r}\right)} \left(\because f(x)\delta(x)= f(0)\delta(x)\right)
\end{equation}
Total Charge, Q:
\begin{gather}
    Q = \int_{-\infty}^{+\infty} \rho d\tau\\
    Q = \int_{-\infty}^{+\infty} A\epsilon_o \left( 4 \pi \delta^3(r) -  \lambda^2\frac{e^{-\lambda r}}{r}\right) d\tau \\
    Q = A\epsilon_o \int_{-\infty}^{+\infty}4 \pi \delta^3(r) d\tau - A \epsilon_{o} \int_{-\infty}^{+\infty} \lambda^2\frac{e^{-\lambda r} }{r} d\tau \\
    Q = A \epsilon_{o} (4 \pi) - A \epsilon_{o} \lambda^2 4 \pi(\frac{1}{\lambda^2})\\
    \boxed{Q = 0}
\end{gather}
\subsection*{Answers}
\(\boxed{\vec{E} = A \frac{e^{-\lambda r}}{r^2} (1 + \lambda r)\hat{r}}\), \\ \(\boxed{\rho = A\epsilon_o \left( 4 \pi \delta^3(r) -  \lambda^2\frac{e^{-\lambda r}}{r}\right)}\) ,\\\(
    \boxed{Q = 0}\).
\newpage
\section{Dipole}
Considering Proton above Z=0 and electron below Z=0, \(10^{-11}m \ll 13m  \), so we can consider this charge distribution as dipole with dipole moment \(\vec{p} = ed \hat{k} \) 
\begin{gather}
    V(R) = \frac{q}{4 \pi \epsilon_{o} }\left(\frac{1}{R_+}-\frac{1}{R_-}\right)
\end{gather}
Law of cosines, 
\begin{gather}
    R_\pm^2 = R^2 + (\frac{d}{2})^2 \mp Rd\cos \phi = R^2(1\mp \frac{d}{R}\cos\phi + \frac{d^2}{4 R^2})\\
    \frac{1}{R_\pm} \approx \frac{1}{R}\left(1\mp \frac{d}{R}\cos\phi\right)^{-\frac{1}{2}} \approx \frac{1}{R}\left(1\pm \frac{d}{2R}\cos\phi\right)\\
    \implies \frac{1}{R_+} - \frac{1}{R_-} \approx \frac{d}{
        R^2}\cos\phi
\end{gather}

\begin{equation}
    V(R) \cong \frac{1}{4\pi \epsilon_{o} } \frac{qd\cos\phi}{R^2}
\end{equation}
\begin{equation}
    \boxed{V(R) = \frac{1}{4\pi \epsilon_{o} } \frac{\vec{p}.\hat{R}}{R^2}}
\end{equation}
Electric Field, \(\vec{E}(R)\) is
\begin{gather}
    \vec{E}(R) = -\vec{\nabla}V(R)\\
    \vec{E}(R) = -\left(\pdv{}{R}\hat{R} +\frac{1}{R \sin \phi}\pdv{}{\theta}+\frac{1}{R}\pdv{}{\phi}\right)\left(\frac{1}{4\pi \epsilon_{o} }\frac{qd\cos\phi}{R^2}\right)\\
    \vec{E}(R) = \frac{qd}{4\pi \epsilon_{o} R^3} \left(2 \cos\phi \hat{R} + \sin \phi \hat{\phi}  \right)
\end{gather}
\begin{equation}
    \boxed{\vec{E}(R) = \frac{1}{4 \pi \epsilon_{o} R^3}\left(3(\vec{p}.\hat{R})\hat{R} -\vec{p} \right)}
\end{equation}
Here, \(\vec{p} = 10^{-11} e \hat{k} \)Cm  and \(\vec{R} = (3 \hat{i} + 4 \hat{i}+12 \hat{k})m \)  
\begin{gather}
    V(R) = \frac{1}{4\pi \epsilon_{o} } \frac{12 X 10^{-11} e }{13^3}V = 4.65 X 10^{-25} V  \\
    \vec{E} (R)= \frac{1}{4 \pi \epsilon_{o} 13^3}(3(\frac{12}{13}X10^{-11}e)\frac{3 \hat{i} + 4 \hat{j} +12 \hat{k} }{13} - 10^{-11}e \hat{k}   )\\
    \vec{E}(R) = (4.188X10^{-24}\hat{i}+ 5.585X10^{-24} \hat{j}+1.019 X 10^{-23} \hat{k} )NC^{-1}  
\end{gather}
\subsection*{Answers}
\(\boxed{V(R) =4.65 X 10^{-25} V}\\\boxed{\vec{E}(R)=  (4.188X10^{-24}\hat{i}+ 5.585X10^{-24} \hat{j}+1.019 X 10^{-23} \hat{k} )NC^{-1} }\)
\section{Magnetic Boundary}

\begin{gather}
    B_{1n} = B_{2n}\\
    H_{1t} = H_{2t}  \implies \frac{B_{1t}}{\mu_1} =  \frac{B_{2t}}{\mu_2} 
\end{gather}
\subsection*{(i) \(\mathbf{\vec{B_1} = 0.5 \hat{x} -10 \hat{y} (mT)}\) }
Boundary is y = 0 therefore, normal is \(\hat{y} \) 
\begin{gather}
    B_{1n} = B_{2n}\implies B_{2n} = -10mT \\
    \frac{B_{1t}}{\mu_1} = \frac{B_{2t}}{\mu_2} \implies  B_{2t} = \frac{\mu_2}{\mu_1}B_{1t}\\
    B_{2t} = 2.5 T\\
    \text{Angle with interface is} \arctan (B_{2n}/B_{2t} ) = \arctan (-\frac{10*10^{-3} }{2.5}) = 0.114^{\circ}
\end{gather}
\begin{equation}
    \boxed{\vec{B_2} = 2.5T\hat{x}  - 10mT \hat{y} , \text{makes } 0.114^{\circ} \text{ with interface.}}
\end{equation}
\subsection*{(ii) \(\mathbf{\vec{B_2} = 10 \hat{x} + 0.5 \hat{y} (mT)}\) }
Boundary is y = 0 therefore, normal is \(\hat{y} \) 
\begin{gather}
    B_{1n}  = B_{2n} \implies B_{1n} = 0.5 mT\\
    \frac{B_{1t}}{\mu_1} = \frac{B_{2t}}{\mu_2} \implies B_{1t} = \frac{\mu_1}{\mu_2}B_{2t}\\
    B_{1t} = \frac{10}{5000}mT = 2 \mu T\\
    \text{Angle with normal is} \arctan(B_{1t}/B_{1n}) = \arctan(\frac{0.002}{0.5}) = 0.229^{\circ}
\end{gather}
\begin{equation}
    \boxed{\vec{B_1} = 2\mu T\hat{x}  +0.5mT \hat{y} , \text{makes } 0.229^{\circ} \text{ with normal.}}
\end{equation}
\section{Maxwell's equations in scalar form}
\subsection*{Maxwell equations}
\begin{gather}  
   \vec{\nabla}.\vec{D} = \rho\\
   \vec{\nabla}. \vec{B} = 0 \\
   \vec{\nabla}X\vec{E}  = -\pdv{B}{t}\\
   \vec{\nabla}X\vec{H} = \vec{J}+\pdv{D}{t}
\end{gather}
\subsection*{Linear Medium}
\begin{gather}
    \vec{D} = \epsilon \vec{E}\\
    \vec{B} = \mu \vec{H}
\end{gather}
\subsection*{(i) Cartesian coordinates}

\subsubsection*{Scalar equations}
\begin{equation}
    \boxed{\pdv{ E_{x} }{x}+\pdv{ E_{y} }{y}+\pdv{ E_{z} }{z} = \frac{\rho}{\epsilon}}
\end{equation}
\begin{equation}
    \boxed{\pdv{B_{x} }{x}+\pdv{B_{y} }{y}+\pdv{B_{z} }{z} = 0}
\end{equation}
\begin{equation}
    \boxed{\left(\pdv{E_{z} }{y}-\pdv{E_{y} }{z}\right)+\pdv{B_{x} }{t}=0}
\end{equation}
\begin{equation}
    \boxed{ \left(\pdv{E_{x} }{z}-\pdv{E_{z} }{x}\right)+\pdv{B_{y} }{t}=0}
\end{equation}
\begin{equation}
    \boxed{\left(\pdv{E_{y} }{x}-\pdv{E_{x} }{y}\right)+\pdv{B_{z} }{t}=0}
\end{equation}
\begin{equation}
    \boxed{\left(\pdv{B_{z} }{y}-\pdv{B_{y} }{z}\right)=\mu J_x + \mu\epsilon \pdv{E_x}{t}}
\end{equation}
\begin{equation}
    \boxed{\left(\pdv{B_{x} }{z}-\pdv{B_{z} }{x}\right)= \mu J_y + \mu\epsilon \pdv{E_y}{t}}
\end{equation}
\begin{equation}
    \boxed{\left(\pdv{B_{y} }{x}-\pdv{B_{x} }{y}\right)=\mu J_z + \mu\epsilon \pdv{E_z}{t} }
\end{equation}
\subsection*{(ii) Cylindrical Coordinates}

\subsubsection*{Scalar Equations}
\begin{equation}
\boxed{\frac{1}{r}\pdv{(rE_r)}{r}+\frac{1}{r}\pdv{E_\phi}{\phi}+\pdv{E_{z} }{z} = \frac{\rho}{\epsilon }}
\end{equation}
\begin{equation}
\boxed{\frac{1}{r}\pdv{(rB_r)}{r}+\frac{1}{r}\pdv{B_\phi}{\phi}+\pdv{B_z}{z} = 0}
\end{equation}
\begin{equation}
\boxed{\frac{1}{r}\pdv{E_z}{\phi}-\pdv{E_\phi}{z}+\pdv{B_r}{t} =0}
\end{equation}
\begin{equation}
\boxed{ \pdv{E_r}{z}-\pdv{E_z}{r}+\pdv{B_\phi}{t}=0 }
\end{equation}
\begin{equation}
\boxed{\pdv{(rE_\phi)}{r}-\pdv{E_r}{\phi}+\pdv{B_z}{t}=0}
\end{equation}
\begin{equation}
\boxed{\frac{1}{r}\pdv{B_z}{\phi}-\pdv{B_\phi}{z}=\mu J_r + \mu \epsilon \pdv{E_r}{t}}
\end{equation}
\begin{equation}
\boxed{\pdv{B_r}{z}-\pdv{B_z}{r}=\mu J_\phi + \mu \epsilon \pdv{E_\phi}{t}}
\end{equation}
\begin{equation}
\boxed{\pdv{(rB_\phi)}{r}-\pdv{B_r}{\phi}=\mu J_z + \mu \epsilon \pdv{E_z}{t}}
\end{equation}
\subsection*{(iii) Spherical coordinates}

\subsection*{Scalar equations}
\begin{equation}
    \boxed{\frac{1}{r^2}\pdv{(r^2 E_r)}{r}+\frac{1}{r \sin \theta}\pdv{E_\theta \sin \theta}{\theta}+\frac{1}{r \sin \theta}\pdv{E_\phi}{\phi}=\frac{\rho}{\epsilon }}
\end{equation}
\begin{equation}
    \boxed{\frac{1}{r^2}\pdv{(r^2 B_r)}{r}+\frac{1}{r \sin \theta}\pdv{B_\theta \sin \theta}{\theta}+\frac{1}{r \sin \theta}\pdv{B_\phi}{\phi}=0}
\end{equation}
\begin{equation}
    \boxed{ \frac{1}{r \sin \theta}\left(\pdv{E_\phi \sin \theta}{\theta}-\pdv{E_\theta}{\phi}\right)+\pdv{B_r}{t}=0}
\end{equation}
\begin{equation}
    \boxed{\frac{1}{r}\left(\frac{1}{\sin \theta}\pdv{E_r}{\phi}-\pdv{(rE_\phi)}{r}\right)+\pdv{B_\theta}{t}=0}
\end{equation}
\begin{equation}
    \boxed{\frac{1}{r}\left(\pdv{(rE_\theta)}{r}-\pdv{E_r}{\theta}\right)+\pdv{B_\phi}{t}=0}
\end{equation}
\begin{equation}
    \boxed{\frac{1}{r \sin \theta}\left(\pdv{B_\phi \sin \theta}{\theta}-\pdv{B_\theta}{\phi}\right)= \mu J_r + \mu \epsilon \pdv{E_r}{t}}
\end{equation}
\begin{equation}
    \boxed{\frac{1}{r}\left(\frac{1}{\sin \theta}\pdv{B_r}{\phi}-\pdv{(rB_\phi)}{r}\right)=\mu J_\theta + \mu \epsilon \pdv{E_\theta}{t}}
\end{equation}
\begin{equation}
    \boxed{\frac{1}{r}\left(\pdv{(rB_\theta)}{r}-\pdv{B_r}{\theta}\right) = \mu J_\phi + \mu \epsilon \pdv{E_\phi}{t}}
\end{equation}
\section{Lorentz Condition and Equation of Continuity}
A \textbf{corollary}  of \textbf{Helmholtz} \textbf{Decomposition} \textbf{theorem} says that all physically realistic scalar fields obey a continuity equation. The theorem states that for any reasonable scalar field S and Vector field \textbf{C}  there exists a vector field \textbf{F} such that \(\vec{\nabla}\).\textbf{F} = S and\(\vec{\nabla}\)X\textbf{F} = \textbf{C}. \href{http://dfcd.net/articles/potentialfields.pdf}{reference}\newline
\textbf{Lorentz Gauge}: 
\begin{gather}
    \vec{\nabla}.\vec{A} = -\frac{1}{c^2}\pdv{\phi}{t}
\end{gather}
from definition  of Magnetic Vector potential, \(\vec{A}\)
\begin{equation}
    \vec{\nabla}X\vec{A} = \vec{B}
\end{equation}
Considering \textbf{F} = $\vec{A}$, S = $-\frac{1}{c^2}\pdv{\phi}{t}$, \textbf{C} = \(\vec{B}\) \\
Lorentz condition satisfy the condition for equation of continuity. 
\section{Homogenous wave equation}
\begin{gather}
    U = f(t \pm R\sqrt{\mu \epsilon } )
\end{gather}
Let \(x = t \pm R\sqrt{\mu \epsilon }\)
\begin{gather}
    \pdv{U}{R} = \odv{f}{x} \pdv{x}{R}\\
    \pdv{U}{R} = \pm\sqrt{\mu \epsilon} \odv{f}{x}
\end{gather}
\begin{equation}
    \boxed{\frac{\partial ^2 U}{\partial R^2} = \mu \epsilon \frac{d^{2}f }{dx^2}}
\end{equation}
\begin{gather}
    \pdv{U}{t} = \odv{f}{x} \pdv{x}{t}\\
    \pdv{U}{t} = \odv{f}{x}
\end{gather}
\begin{equation}
    \boxed{\frac{\partial ^2 U}{\partial t^2} = \frac{d^{2}f }{dx^2}}
\end{equation}
\begin{equation}
    \boxed{\frac{\partial ^2 U}{\partial R^2} -\mu \epsilon \frac{\partial ^2 U}{\partial t^2} = \mu \epsilon \frac{d^{2}f }{dx^2} - \mu \epsilon \frac{d^2 f}{dx^2} = 0}
\end{equation}
Therefore, any function of $t\pm R \sqrt{\mu \epsilon}$ satisfies the Homogenous wave equation.  
\end{document} 
