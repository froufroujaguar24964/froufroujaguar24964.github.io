\lecture{3}{27 July 2022}{Third Lecture}
\subsection*{Boundary Conditions}
\subsection*{Normal Component}
\begin{figure}[H]
    \centering
    \incfig{normcom}
    \caption{title}
    \label{fig:normcom}
\end{figure}
\[
    \oint \vec{B}. d\vec{s} = 0 
\]
\[
    B_{1n} \Delta S - B_{2n} \Delta S = 0 \implies B_{1n} = B_{2n} 
\]
\[
    \oint \vec{H}.d\vec{l} = I_{f} \vspace{0.3pt}
\]
\subsubsection*{Tangential Component}
\begin{figure}[H]
    \centering
    \incfig{magtan}
    \caption{title}
    \label{fig:magtan}
\end{figure}
\[
    H_{1t}\Delta W - H_{2t} \Delta W = \Delta W k_s 
\]
\[
    H_{1t} - H_{2t}=k_s 
\]Where, \(k_{s} \) = surface current density. 
\[
    \bar{n_2} X (\vec{H_1} -\vec{H_2}) = \vec{k_s}
\] 
\subsection*{Ohms law}
\[
    \vec{J} = \sigma \vec{E}
\]
\[
    \epsilon = -\frac{dQ}{dt} = \oint \vec{E}.d\vec{l}
\]
Faraday's laws of induction
\[
    \vec{\nabla}X\vec{E} = -\frac{\partial \vec{B}}{\partial t}
\]
Energy stored in a Magnetic Field
\[
    W = \frac{1}{2 \mu_{o}} \int B^2 d\tau 
\]
\subsection*{Summary of Equations}
Gauss law
\[
    \vec{\nabla}. \vec{D} = \rho_{f} 
\]
Magnetic monopole doesn't exist
\[
    \vec{\nabla}.\vec{B} = 0
\]
Faraday's law
\[
    \vec{\nabla}X\vec{E} = - \frac{\partial \vec{B}}{\partial t}
\]
Ampere's law
\[
    \vec{\nabla}X\vec{H} = J_{f} 
\]
\subsection{Correction in Ampere's law}
\[
    \vec{\nabla}.\vec{\nabla}X\vec{H} = 0, 
\]
but 
\[
    \vec{\nabla}.\vec{J_{f} } \neq 0
\]
\[
    \vec{\nabla}.\vec{J} = -\frac{\partial \rho}{\partial t}
\]
\[
    \vec{\nabla}X\vec{H} = \vec{J_{f} }+ ?
\]
\[
    \vec{\nabla}.\vec{\nabla}X\vec{H} = \vec{\nabla}.\vec{J_{f} }+ \frac{\partial \rho_{f} }{\partial t}
\]
\[
    \vec{\nabla}.\vec{\nabla}X\vec{H} = \vec{\nabla}.\vec{J_{f} }+\frac{\partial \vec{\nabla}.\vec{D}}{\partial t}
\]
\[
    \vec{\nabla}.\vec{\nabla}X\vec{H} = \vec{\nabla}.\left(\vec{J}+\frac{\partial \vec{D}}{\partial t}\right)
\]
\begin{definition}
    Displacement current density \(\frac{\partial \vec{D}}{\partial t}\) 
\end{definition}
\[
    \vec{D} = \epsilon \vec{E}
\]
\[
    \vec{B} = \mu \vec{H}
\]

\subsection*{Non-linear dielectric}

\[
    \vec{P_{i}} = \Sigma_j \alpha_{ij} E_j + \Sigma_{jk} \alpha_{ijk} E_{j} E_{k}  
\]
\subsection*{Linear dielectric}
\[
    \vec{P_{i} } = \epsilon_{o} \chi_{ij}\vec{E_{jk} } 
\]
\begin{eg}
    \[
        P_{x} = \epsilon_o (\chi_{xx}{E_{x}}+\chi_{xy}E_{y} +\chi_{xz}E_{z})
    \]
    \[
        P_y = \epsilon_{o} (\chi_{yx}E_{x} +\chi_{yy}E_{y}+\chi_{yz}E_{z}  )
    \]
    \[
        P_{z} = \epsilon_o(\chi_{zx}E_x + \chi_{zy}E_{y} +\chi_{zz} E_{z} )
    \]

\end{eg}
\subsection*{Linear Isotropic Dielectric}
\(\chi \to \) independent of direction of electric field. 
\subsection*{Linear Homogenous Isotropic Dielectric}
\(\chi \to \) Same everywhere. 
\subsection*{Potential formulation}
\[
    \vec{B} = \vec{\nabla}X\vec{A} \left(\because \vec{\nabla}.\vec{B}=0\right)
\]
Faraday's law
\[
  \vec{\nabla}X\vec{E} = \frac{\partial \vec{B}}{\partial t}  = \frac{\partial (\vec{\nabla}X\vec{A})}{\partial t}
\]
\[
    \vec{\nabla} X \vec{E} = -\vec{\nabla}X\frac{\partial \vec{A}}{\partial t }
\]
\[
    \vec{\nabla} X \left(\vec{E} + \frac{\partial \vec{A}}{\partial t}\right) = 0
\]
\[
    \vec{E} + \frac{\partial \vec{A}}{\partial t} = - \vec{\nabla}V
\]
\[
    \vec{E} = -\vec{\nabla}V - \frac{\partial \vec{A}}{\partial t}
\]
\[
    \vec{B} = \vec{\nabla}X\vec{A}
\]
Solution for time-independent can be given by poisson's equations, but for time dependent we cannot use poisson's equations. 

\textbf{Conditions} 
\begin{enumerate}
    \item At low frequency :- changes very slowly
    \item Near to source :- quasi-static condition
\end{enumerate}
